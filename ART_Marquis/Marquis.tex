\begin{artengenv}{Jean-Pierre Marquis}
	{Abstract logical structuralism\edtfootnote{The author gratefully acknowledges the financial support of the SSHRC of Canada while this work was done. I also want to thank the organizers of the 23\textsuperscript{rd} Kraków Methodological Conference for inviting me and giving me the opportunity to give a talk in such a lovely venue.}}
	{Abstract logical structuralism}
	{Abstract logical structuralism}
	{Department of Philosophy, University of Montreal}
	{Structuralism has recently moved center stage in philosophy of mathematics. One of the issues discussed is the underlying logic of mathematical structuralism. In this paper, I want to look at the dual question, namely the underlying structures of logic. Indeed, from a mathematical structuralist standpoint, it makes perfect sense to try to identify the abstract structures underlying logic. We claim that one answer to this question is provided by categorical logic. In fact, we claim that the latter can be seen---and probably should be seen---as being a structuralist approach to logic and it is from this angle that categorical logic is best understood.}
	{none.}





\section{Introduction}
\lettrine[loversize=0.13,lines=2,lraise=-0.03,nindent=0em,findent=0.2pt]%
{I}{}n their recent booklet \emph{Mathematical Structuralism}, \citeauthor{HellmanShap2019} \parencite*{HellmanShap2019}, %Hellman \& Shapiro 
give a list of eight criteria to evaluate various strands of mathematical structuralism. The very first criterion is the background logic used to express the version of structuralism examined: is it first-order, second-order, higher-order, modal? The claim made by Hellman \& Shapiro is that the logic has a direct impact on the philosophical thesis: for instance, the existence of non-standard models in first-order logic seems to affect some of the claims made by a mathematical structuralist. Be that as it may, it is assumed that any variant of mathematical structuralism is based on an underlying logic and the chances are that a change of logic will modify the type of structuralism defended. One then has to weigh the pros and the cons of adopting a specific logic to defend a kind of mathematical structuralism. 

This is all well and good, and I don't intend to discuss this assumption in this paper. Let me rather turn the question of the underlying logic on its head. A mathematical structuralist not only believes that \emph{pure} mathematics is about abstract structures, but also that \emph{pure} logic can be seen that way. In other words, a mathematical structuralist aspires to know what are the abstract structures underlying a given logic. Can logic itself  be given a structuralist treatment? In other words, is it possible to identify the \emph{abstract mathematical structures} from which the standard logical systems can be derived? In the same way that the natural numbers or the real numbers can be seen as specific structures arising from the combination of particular structures and properties, specific logical systems, e.g. intuitionistic first-order logic, classical first-order logic would result from the combination of particular abstract structures and properties. And important logical theorems---completeness theorems, incompleteness theorems, definability, undecidability, etc.---would be instances of abstract structural features, together with, presumably, singular properties. 

Our main claim in this paper is that categorical logic is one way to give precise answers to these questions.\footnote{In his \parencite*{Awodey1996}, often quoted as representing the categorical perspective on mathematical structuralism, Awodey did include a presentation of the basic features of logic from a categorical point of view, more specifically the internal logic of a topos. Given how he treated logic in his paper, none of his critics saw that he was including logic itself as an object of mathematical structuralism  \parencite[see, for instance,][]{Hellman2001,Hellman2003}. We approach the issues from a different angle.} Indeed, this is what categorical logic is all about: it reveals the abstract mathematical structures of logic and it relates them to other abstract mathematical structures, revealing yet other structural features. It also shows how the main results of logic are a combination of abstract structural facts and specific properties of logical systems. It is in this sense that one can say that \emph{pure} category theory \emph{is} logic.\footnote{Which is not to say that it is \emph{all} of logic. That is \emph{not} the claim.} This is one of the main \emph{points} of categorical logic. Thus, when mathematical structuralism is developed within the metamathematical framework of category theory, it is possible to give a positive answer to our new challenge. And as far as I know, it is at the time the only metamathematical framework that allows us to do it in such generality.

Of course, we are not claiming that the abstract analysis is \emph{superior} in a strong sense to the other presentations of logical systems. It brings a certain perspective, a certain understanding and opens up certain connections that are otherwise unavailable. Furthermore, in as much as a logic is applied, be it in foundational studies or in computer science, one also wants to look at it from a different point of view. But it has to be perfectly clear that these are completely different issues. We are now positioning ourselves in a structuralist framework, and will therefore ignore the aspects of a logic that become prevalent when one look at it for its applications. Ours is a \emph{philosophical} goal, not a technical one.


It goes without saying that categorical logic, seen as a search for the abstract structures of logic, did not come out of the blue. It was part of a larger mathematical movement, namely the structuralist movement in mathematics, which culminated in Bourbaki's \emph{Éléments de mathématique} \parencite[for more on the prehistory and the history of mathematical structuralism, see][]{Corry2004, Reck:2020}. It will therefore be worth our while to look briefly at the birth of categorical logic in the 1960s and early 1970s. We will merely indicate the major landmarks of the story, to see how indeed categorical logic was, right from the start, understood as being a part of that movement. We will then briefly look at some of the basic abstract structures that correspond to logic and then how some of the main theorems follow from features of abstract structures. At this point, to call the latter `mathematical' or `logical' is a matter of choice.

In the end, this paper should be taken as a challenge. We claim that one should add a ninth criterion of Hellman \& Shapiro's list: how does a given form of structuralism treat logic itself? Does it reveal the structures of logic? Are these structures related to other mathematical structures in a natural manner? Are these structures on the same level as the other fundamental structures of mathematics? One could, and we suggest that one should, evaluate different forms of mathematical structuralism according to this criterion. 

\section{The abstract structures of logic:\\setting the stage}

Although category, functors and natural transformations were introduced explicitly  in \cite*{EilMac1945} by \citeauthor{EilMac1945}, category \emph{theory} came to maturity only fifteen years later, thus at the beginning of the 1960s  \parencite[for more on this history, see, for instance][]{LandryMarquis2005,Kromer2007,Marquis2009,Rodin2014}. At that point, the major concepts of category theory were in place, e.g. adjoint functors, representable functors, constructions on categories, including the crucial construction of functor categories, abstract categories like additive categories, abelian categories, tensor or monoidal categories, etc.\footnote{The \emph{locus classicus} of the time is \parencite{MacLane1965}.} But it is only in the early 1960s that the connection with logic and the foundations of mathematics was made and it was mostly the work of one person, namely Bill Lawvere. We will not present Lawvere's early work here, for our goal is not to survey the history of the subject, but rather to make a conceptual point.\footnote{See \parencite{Lawvere1963,Lawvere1966a,Lawvere1967,Lawvere1969a,Lawvere1970,Lawvere1971}. He was rapidly joined by \parencite{Freyd1966,Linton1966,Lambek1968a,Lambek1968b,Lambek1969,Lambek1972}. For a detailed history, see \parencite{Marquis2012}.} 

Of course, by that time, connections between classical propositional logic and Boolean algebras, intuitionistic propositional logic and Heyting algebras, as well as others were known. Already in \parencite{Birkhoff1940}, the main relations are presented.\footnote{We could argue  that already for propositional logics, the structures arising from the logical systems naturally live in categories, e.g. the category of distributive lattices, the category of Boolean algebras, the category of Heyting algebras, the category of S4-algebras, etc., and that the main results are also naturally expressed in categorical language. It is in fact an important point to make, for once the higher order logical systems find their place in this landscape, the fact that we move to bicategories to express and prove the results of first-order logic is easily understood. But we will not dwell on that.}  The first links between logic and lattice theory were restricted to propositional logic. The chase for the identification of the abstract structures corresponding to first-order, higher order logics, non-classical logics, as well as algebraic proofs of the main theorems of logic was taken up by, among others, Tarski and his school, Halmos, and the Polish school, e.g. \L o\'{s}, Mostowski, Rasiowa and Sikorski.\footnote{The list of references is long, but clearly indicates that it was a very active area of research in the 1950s as well as in the 1960s. See, for instance,
 \parencite{McKinseyTarski1944,McKinseyTarski1946,McKinseyTarski1948,JonssonTarski1951,JonssonTarski1952,HenkinTarski1961,HenkinTarski1971,Halmos1954,Halmos1956,Halmos1956a,Halmos1956b,Halmos1956c,Halmos1962,Mostowski1949,Rasiowa1951,Rasiowa1955,RasiowaSik1950,RasiowaSik1953,RasiowaSik1955,RasiowaSik1963}.} 
The main contenders to the title of abstract algebraic structures corresponding to first order logic at the time were cylindric algebras and polyadic algebras. The main problem, so to speak, were the quantifiers $\forall$ and $\exists$. It was not a technical problem. They were treated properly in each case. But their treatment, as algebraic operators, was somewhat \emph{ad hoc}, in the sense that they did not arise as an instance of abstract operators in an algebraic context. The resulting algebras were therefore somewhat \emph{ad hoc} also, in as much as they did not belong to a family of abstract structures that arose naturally in other mathematical contexts. In other words, the abstraction proposed via the concepts of cylindric algebras or polyadic algebras were not genuine mathematical abstractions, for they were merely the algebraic transcription of the quantifiers and solely of the latter.\footnote{The reader might wonder what we mean by ``genuine mathematical abstraction''. We refer her to \parencite{Marquis2014,Marquis2016}.} This is in stark contrast with the case of propositional logic, where the abstract algebraic structures capturing the logic and its main properties have instances in a variety of completely different mathematical fields. Distributive lattices, Boolean algebras, Heyting algebras, etc., are genuine mathematical abstractions.

It therefore came as a complete surprise that the quantifiers, as well as the propositional connectives, could be seen as being instances of adjoint functors on very simple categories, the concept of adjoint functors being one of the core concepts of category theory, introduced by Kan in the context of algebraic topology in 1958. This was one of Lawvere's crucial observations. Three additional crucial facts had been established by Lawvere in his Ph.D. thesis. First,   Lawvere showed how  algebraic theories, in the standard logical sense of that expression, could themselves be captured by specific categories. Second, the models of algebraic theories, again in the standard logical sense of that expression, could be described in the language of categories, functors and natural transformations. Third, the classical links between the syntax and the semantics of these theories could receive an adequate categorical treatment, and at the core of this treatment one finds adjointness. Thus, it seemed possible to put all the structures of logic in the theoretical framework of categories, the latter being, of course, an abstraction of a central fact of modern mathematics. The overall plan was presented by Lawevere in \parencite*{Lawvere1969a}. That paper articulates in very broad strokes how the syntax, the semantics and their relationships could be captured in a categorical framework. Here are, in a nutshell, the main ingredients of this ambitious program.

A few words about the philosophical framework underlying Lawvere's program are in order. Lawvere identifies two fundamental aspects to all of mathematics, namely the formal and the conceptual, roughly the manipulation of symbols, on the one hand, and what these symbols refer to, their content. Lawvere is aware of the work done in algebraic logic when he writes his paper. Indeed, he refers to it explicitly in the opening section: ``[...] Foundations may conceptualize the formal aspect of mathematics, leading to Boolean algebras, cylindric and polyadic algebras, [...]''\parencite[][p.281]{Lawvere1969a} He is also presenting the introduction of categories in the analysis of logic as a structural approach, based on the notion of adjoints: ``Specifically, we describe [...] the notion of cartesian closed category, which appears to be the appropriate abstract structure for making explicit [...]. The structure of a cartesian closed category is entirely given by adjointness, as is the structure of a `hyperdoctrine', which includes quantifiers as well.''\parencite[][p.281]{Lawvere1969a} 

We will now focus on the final section of the paper, which is really programmatic. In this last section, Lawvere is describing what he himself characterizes as a globalized Galois connection, and indeed, it also contains the main ideas that have driven the development of duality theory in a categorical framework. But as far as logic is concerned, we are offered the following picture.
\begin{enumerate}
	\item Logical operations should arise from an elementary context as adjoint operations. From a categorical point of view, a logical doctrine, that is an abstract mathematical structure encapsulating a logical framework, should be given by adjoint functors.\footnote{We have to point out that categorical logic does cover logical situations in which certain logical operations are not given by adjoint operations. Although they do not constitute logical \emph{doctrines} in the sense of Lawvere, they are part and parcel of categorical logic. We should also mention that the syntactic aspects of logic, which are pushed in the background in Lawvere's early work, occupy nonetheless an important part of categorical logic, for instance via the notion of sketch, introduced by Ehresmann and his school or the various formal graphical languages developed mostly in the context of monoidal categories.}
	\item A theory $\mathbf{T}$, in the standard logical sense of the term, should be constructed as a category, in the same way that a propositional theory in classical logic can be turned into a Boolean algebra via the Lindenbaum-Tarski construction. A theory $\mathbf{T}$, seen as a special type of category, is conceived by Lawvere as being the invariant notion of a theory, that is, independent of a choice of primitive symbols or specific axioms. We thus have abstract mathematical structures corresponding to the formal.
	\item The models of $\mathbf{T}$ should form a category. Lawvere, having himself developed the case of algebraic theories earlier in his thesis, generalizes from his work and proposes to make the category of models of a theory a functor category. We will be more specific in later sections. These provide the abstract mathematical structures emerging from the conceptual.
	\item Last, but certainly not least, since everything is a category now, the links between the formal and the conceptual should also be given by (adjoint) functors, and we have yet again a new type of abstract mathematical structure, in Lawvere's mind a globalized Galois connection, arising from that situation.
\end{enumerate}
Lawvere was of course guided by his own work on algebraic theory, but also explicitly by Grothendieck's work in algebraic geometry. As I said, at the time, it was a program, some would say a vision. It became a reality in the following decade and is still the basis of important developments in the field. 

We have to explain why we claim that we are then in a structuralist framework. It is not only because we are in fact dealing with abstract mathematical structures---this is of course a necessary step---but these abstract mathematical structures can be characterized up to `isomorphism', where the latter notion is derived from the abstract structures themselves. Each and every one of these abstract structures comes with a notion of homomorphism and, in particular, a notion of `isomorphism'. Therefore, it becomes possible to develop logic with respect to the structuralist principle: if $X$ is a structure of a given kind, and it has property $P(X)$, then given any other structure $Y$ of the same kind such that $X$ is isomorphic to $Y$, $X \simeq Y$, in the appropriate sense of isomorphism, we should be able to prove that $P(Y)$. As it can be seen, the key component of this desideratum is the appropriate sense of isomorphism. In some cases, we are dealing with the usual set-theoretical sense of isomorphism, in others, it becomes an equivalence of categories and in still others, it is a 2-categorical equivalence. It is the very possibility of having the appropriate sense of isomorphism that allows us to claim that we are dealing with \emph{abstract} mathematical structures. 

I want to emphasize again, at this point, that I am not claiming that categorical logic, as I will present it succinctly below, is the only possible answer for a structuralist nor is it the final answer. But it is one clear answer and one of the very few that provides a comprehensive answer. 

\section{Categories as abstract logical structures}

At this stage, we would have to give a long list of definitions and examples to illustrate how certain categories correspond to the abstract mathematical structures of certain logics. I will assume that the reader knows the notions of category, functor, natural transformation, adjoint functors, etc., for otherwise this paper would be terribly long and boring. We will try to put some flesh on Lawvere's program described in the foregoing section. We assume, however, that the description of the logical connectives, including the quantifiers, as adjoint functors is understood\footnote{Mac Lane's textbook, \parencite*{MacLane1998}, is still a good reference. All the standard concepts and examples can also be found in \parencite{Riehl2017}.}. We will sketch how the other three steps are filled\footnote{A more detailed presentation can be found in \parencite{Marquis2009}, chapter 6. Our exposition here is adapted from the latter, but the philosophical point is different and therefore the presentation found there might not be optimal for our present purpose.}. A warning is necessary. Each following section would require a careful and systematic exposition to be ultimately convincing. It is impossible to do justice to the field in such a short paper. We will provide a more detailed presentation in the next section only and merely gloss over the abstract mathematical structures involved in the other sections. We apologize for the opacity that might result from the lack of details and clarifications, but a much longer paper would be required to present and motivate adequately the main mathematical ideas involved. 

\subsection{A theory as a category and a category as a theory}

Let us start with the goal that was in the minds of logicians and mathematicians in the 1950s, that is finding the appropriate abstract mathematical structure that correspond to a first-order theory. 

Let us fix the logical context first. We consider formal systems with many sorts, which is a simple generalization of the standard first-order logic which is done over a single sort. A \textit{similarity type} or \textit{alphabet} \( A \), often called a language in the literature, is given by:
%
\begin{enumerate}

\item A collection of sorts \( S_1, S_2, S_3, \dots \);\footnote{Or types, if you prefer. We are dealing here with first-order logic. We will say a few words about type theory later. It does not affect our basic general point. Type theories can also be analyzed as instances of abstract mathematical structures.}

\item A collection of relation symbols \( R_1, R_2, R_3, \dots \), each of which is given with the sorts of its arguments;

\item A collection of function symbols \( f_1, f_2, f_3, \dots \) each of which is given with the sorts of its arguments and the sort of its target; we denote a function symbol \( f \) as \( f\colon S_1 \times \dots \times S_n \rightarrow S \) if \( f \) takes \( n \) arguments of sorts \( S_1, \dots, S_n \) 
respectively to a value of sort \( S \).

\item A collection of constants \( c_1, c_2, c_3, \dots \) each with a specified sort; we denote a constant \( c \) by \( c\colon 1 \rightarrow S_i \) to indicate that the constant \( c \) is of sort \( S_{i} \).
\end{enumerate}
%
This is the standard definition extended to a many-sorted context. To obtain a \textit{formal system \( L_{A} \)}\index{Formal system} in the alphabet \( A \), we add the usual elements:
%
\begin{enumerate}

\item Each sort \( S_i \) comes with infinitely many variables \( x_1, x_2, x_3, \dots \); we write \( x\colon S_i \) to indicate that the variable \( x \) is of sort \( S_i \);

\item Each sort has an equality relation \( =_{S} \); notice immediately that this means that equality is not treated as a universal or purely logical relation and that in the interpretation, 
whatever will correspond to a sort will have to come equipped with a criterion of identity or equality for \textit{its} objects;

\item The usual logical symbols and two propositional constants, \( \top \) and \( \bot \);

\item The usual deductive machinery for a predicate logic (say, intuitionistic predicate logic; if any other deductive procedures are assumed, they are made explicit).
\end{enumerate}
%
Terms (of a given sort) and atomic formulas are defined as usual. 


Here is the first original result obtained by searching for abstract mathematical structures corresponding to theories in a given logic. Some fragments of first-order logic and some extensions of first-order logic turn out to have significant properties, properties that would not have been identified otherwise\footnote{We will not be exhaustive here. There are other infinitary fragments that are important, but we will ignore them.}. We can immediately identify  the following fragments.
\begin{enumerate}
	\item A formula \( \varphi \) is said to be \textit{regular} if it is obtained from atomic formulas by applying finite conjunction and existential quantification.
 	\item A formula \( \varphi \) is said to be \textit{coherent} if it is obtained from atomic formulas by applying finite conjunction, disjunction and existential quantification. 
	\item A formula is said to be \textit{geometric}  if it is obtained from atomic formulas by applying \textit{finite} conjunction, \textit{finite} existential quantification and \textit{infinite} disjunction. 
\end{enumerate}

Intuitionistic formulas and classical (or Boolean) formulas are defined in the obvious manner. We point out immediately that many metalogical results about intuitionistic and classical logic follow directly from results about the foregoing fragments. This is a genuine discovery that could not have been foreseen beforehand. 

An \textit{implication} of regular (resp. coherent, geometric, etc.) formulas \( \varphi \) and \( \psi \) has the form
%
\[
\forall x_1 \dots \forall x_n (\varphi(x_1, \dots, x_n) \Rightarrow \psi(x_1, \dots, x_n))
\]
%
where \( \varphi(x_1, \dots, x_n) \) and \( \psi(x_1, \dots, x_n) \) are regular (resp. coherent, geometric, etc.) formulas. A theory \( \mathbf{T} \) in the given language \( L \) is said to be a \textit{regular theory} (resp. coherent, geometric, etc.) if all its axioms are implications of regular (resp. coherent, geometric, etc.) formulas. Many mathematical theories can be expressed in the form of regular theories, or coherent theories, etc. 


Apart from the fact that we have assumed a many sorted context and cut the fragments of first-order logic in ways that might seem arbitrary, the foregoing presentation is squarely in a standard logical context. We now move to the abstract mathematical context.

Given a  theory \( \mathbf{T} \) in one of the foregoing languages, we construct a category denoted by \( [\mathbf{T}] \), out of it.  The latter is sometimes called the category of concepts, sometimes the syntactic category, and it is basically an extension of the Lindenbaum-Tarski construction, but for theories expressed in first-order logic (and others, as the reader can now guess). It is constructed from the language and the axioms of \( \mathbf{T} \) as follows.\footnote{See also \parencite[][chap.~8]{MakkaiReyes1977} or \parencite[][chap.~X, \S~5]{MacLane1994} for more details and proofs or again \parencite{Johnstone2002}.} 


Remember that we are constructing a category, thus a web of objects connected by morphisms. To get the objects of this category, we start with \textit{formal sets} \( [\lvec{x}; \varphi(\lvec{x})] \), where \( \lvec{x} \) denotes a \( n \)-tuple of distinct variables containing all free variables of \( \varphi \) and \( \varphi \) is a formula of the underlying formal system \( L \). Two such formal sets, \( [\lvec{x}; \varphi(\lvec{x})] \) and \( [\lvec{y}; \varphi(\lvec{y})] \) are equivalent if one is the alphabetic variant of the other, that is if \( \lvec{x} \) and \( \lvec{y} \) have the same length and sorts and \( \varphi(\lvec{y}) \) is obtained from \( \varphi(\lvec{x}) \) by substituting \( \lvec{y} \) for \( \lvec{x} \) (and changing bound variables if necessary). This is an equivalence relation and it is therefore possible to consider equivalence classes of such formal sets. An object of the category of concepts \( [\mathbf{T}] \) is such an equivalence class of formal sets \( [\lvec{x}; \varphi(\lvec{x})] \), where \( \varphi \) is a formula of the formal system \( L \). The objects of \( [\mathbf{T}] \) are the \emph{equivalence classes of these formal sets}, for all formulas of \( L \). Notice this last important point: we take \textit{all} formulas of the language, not only those which appear in \( \mathbf{T} \). Thus, in a sense, the space of objects is the collection of all possible properties and sentences expressible in that language, thus all possible theories in the given formal system. No logical relationship is considered at this stage. The next step introduces the structure corresponding to the structure of that particular theory \( \mathbf{T} \). This is just as one would expect in a categorical framework: the structure of \( \mathbf{T} \) is captured by the morphisms we will define and the properties resulting therefrom.

It is easier to motivate the definition of morphism with an eye on the semantics, although the properties of the morphisms, e.g., that they form a category, have to be proved with the syntactical features of the theory (unless one has a completeness theorem at hand). The basic idea is this: a functor from \( [\mathbf{T}] \) to \( \stcat{Set} \) should transform the objects of \( [\mathbf{T}] \) into \emph{genuine sets} and the morphisms of \( [\mathbf{T}] \) into \emph{genuine functions} automatically. These functions should be functions that are definable in \( \mathbf{T} \)---i.e., for which we can prove in \( \mathbf{T} \) that they are indeed functions. Furthermore, \( [\mathbf{T}] \) should contain all of them. By sending a formal set \( [\lvec{x}; \varphi(\lvec{x})] \) to the set of \( n \)-tuples satisfying the formula, i.e. \( \{ (x_1, \dots, x_n) \mid \varphi(\lvec{x}) \} \), a morphism from \( [\lvec{x}; \varphi(\lvec{x})] \) to \( [\lvec{y}; \psi(\lvec{y})] \) should become a genuine function between genuine sets \( \{ (x_1, \dots, x_n) \mid \varphi(\lvec{x}) \} \) and \( \{ (y_1, \dots, y_m) \mid \psi(\lvec{y}) \} \) respectively. Such a morphism should simply be given by a formula of the theory \( \mathbf{T} \) that defines such a function, that is a formula \( \theta(\lvec{x}, \lvec{y}) \) of \( \mathbf{T} \) that is provably functional. The only trick in the construction is to construct a morphism between two (equivalence classes of) formal sets \( [\lvec{x}; \varphi(\lvec{x})] \) and \( [\lvec{y}; \psi(\lvec{y})] \) in such a way that, when interpreted, it yields the \textit{graph} of the function, in the standard set-theoretical sense of that expression, between the actual sets \( \{ (x_1, \dots, x_n) \mid \varphi(\lvec{x}) \} \) and \( \{ (y_1, \dots, y_m) \mid \psi(\lvec{y}) \} \). Thus, all definable functions in \( \mathbf{T} \) will be represented by a morphism in \( \mathbf{[T]} \).


Formally, consider a triple \( (\lvec{x}, \lvec{y}, \gamma) \), where \( \lvec{x} \) and \( \lvec{y} \) are disjoint tuples of distinct variables and \( \gamma \) is a formula with free variables possibly among \( \lvec{x} \) and \( \lvec{y} \). Such a triple defines a \emph{formal function} if the following formulas are provable:
%
\begin{align*}
\mathbf{T} \vdash \forall \lvec{x} \forall \lvec{y} & (\gamma(\lvec{x}, \lvec{y}) \Rightarrow (\varphi(\lvec{x}) \land \psi(\lvec{y}))) \text{;} \\
\mathbf{T} \vdash \forall \lvec{x} & (\varphi(\lvec{x}) \Rightarrow \exists \lvec{y} (\gamma (\lvec{x}, \lvec{y}))) \text{;} \\
\mathbf{T} \vdash \forall \lvec{x} \forall \lvec{y} \forall \lvec{y}^{\prime} & ( \gamma(\lvec{x}, \lvec{y}) \land \gamma(\lvec{x}, \lvec{y}^{\prime}) \Rightarrow \lvec{y} = \lvec{y}^{\prime}) \text{;}
\end{align*}
%
where we have used some obvious abbreviations. The underlying motivation should be clear: these formulas will be true in any interpretation of \( \mathbf{T} \) in which \( \gamma \) is indeed a morphism.

We now define an equivalence relation \( (\lvec{x}, \lvec{y}, \gamma) \sim (\lvec{u}, \lvec{v}, \eta) \) if 
%
\[
\mathbf{T} \vdash \forall \lvec{x} \forall \lvec{y} (\gamma \Leftrightarrow (\eta(\lvec{x}/\lvec{u}, \lvec{y}/\lvec{v}))) \text{.}
\]
%
The equivalence relation guarantees that for every model \( M \) of \( \mathbf{T} \), the functions corresponding to \( \gamma \) and to \( \eta \) will coincide. We can now stipulate that a formal function is an \emph{equivalence class} of the foregoing equivalence relation. Given a representative \( (\lvec{x}, \lvec{y}, \gamma) \) of such an equivalence class, we denote the equivalence class containing it by \( \langle \lvec{x} \mapsto \lvec{y}\colon \gamma \rangle \). Thus, a \textit{formal morphism} in \( [\mathbf{T}] \) is denoted by:
%
\[
\langle \lvec{x} \mapsto \lvec{y} \rangle \colon [\lvec{x}\colon \varphi] \rightarrow [\lvec{y}\colon \psi].
\]

We need two more ingredients to get to a category. Firstly, for each formal set \( [\lvec{x}\colon \varphi(\lvec{x})] \), the identity morphism is provided by the formal morphism \( \langle \lvec{x} \mapsto \lvec{y} \colon ( \lvec{x} = \lvec{y} ) \land \varphi \rangle \). Secondly, given two formal morphisms \( \langle \lvec{x} \mapsto \lvec{y} \colon \gamma \rangle\colon [\lvec{x}\colon \varphi] \rightarrow [\lvec{y}\colon \psi] \) and \( \langle \lvec{y} \mapsto \mathbf{z} \colon \eta \rangle\colon [\lvec{y}\colon \psi] \rightarrow [\mathbf{z}\colon \zeta] \), their composition is defined by the formal morphism \( \langle \lvec{x} \mapsto \mathbf{z} \colon \mu \rangle\colon [\lvec{x}\colon \varphi] \rightarrow [\mathbf{z}\colon \zeta] \) where \( \mu = \exists \lvec{y} (\gamma \land \eta) \). These two definitions satisfy the usual requirements of a category. Thus, \( [\mathbf{T}] \) is a category and we have an abstract mathematical structure corresponding to a given theory.

Notice that \( [\mathbf{T}] \) is \textit{not} a category of structured sets and structure-preserving functions! A lot of information about \( \mathbf{T} \) is lost when all we have at our disposal is \( [\mathbf{T}] \). It is, for instance, impossible to know which atomic formulas are involved in specific formal sets or what were the primitive symbols of the language \( L_\mathbf{T} \). Furthermore, two different theories \( \mathbf{T} \) and \( \mathbf{T}^\prime \) can very well yield isomorphic categories of concepts, thus essentially the same category. We are squarely in a structuralist framework: the category \( [\mathbf{T}] \) is given up to an isomorphism of categories.\footnote{Notice that we are talking about isomorphism here and not an equivalence of categories.} As we have seen, the syntactic logical operations, i.e., quantifiers and connectives, become categorical operations in the category and this part of the structure is not lost. Again, moving from a theory \( \mathbf{T} \) to its category of concepts \( [\mathbf{T}] \) is an \textit{abstraction}: the  specific formulas with specific variables are abstracted from when we move to the equivalence classes. The category of concepts is the category of all definable sets and functions of a theory \( \mathbf{T} \). Thus, in a sense, it contains all the formally expressible concepts of \( \mathbf{T} \), whence its name. 

When we start with, for example, a regular theory, the foregoing construction yields a category with additional structures and properties. For instance, it is automatically a category with finite limits. 

\( \mathbf{T} \) and \( [\mathbf{T}] \) are interchangeable in the following sense: for, given a (small) category \( \cat{C} \), at least with finite limits, it is possible to associate or construct the language \( L_{\cat{C}} \) of \( \cat{C} \) as follows. We first have to identify the alphabet of \( L_{\cat{C}} \). The sorts are given by the objects \( X, Y, Z, \dots \) of   \( \cat{C} \). Every morphism \( f\colon X \rightarrow Y \) of \( \cat{C} \) becomes a function symbol of \( L_{\cat{C}} \). (In particular, a constant \( c\colon 1 \rightarrow X \) is seen as \( 0 \)-ary function symbols.) This is called the \textit{canonical language}\index{Canonical language of a category} of \( \cat{C} \). Notice that \( L_{\cat{C}} \) is obtained as if we had taken \( \cat{C} \) and destroyed its categorical structure, retaining only the symbols, and keeping in mind that function symbols are sorted. It is possible to extend this language to reflect the structure of \( \cat{C} \) more closely. Although subobjects of \( \cat{C} \) can be denoted naturally by formulas of \( L_{\cat{C}} \), it is possible to introduce relation symbols for each subobject \( R(x_1, \dots, x_n) \rightarrowtail X_1 \times \dots \times X_n \) and \( n \)-ary function symbol for morphisms \( f\colon X_1 \times \dots \times X_n \rightarrow X \). This is called the \textit{extended} canonical language\index{Canonical language of a category!Extended---}of \( \cat{C} \) \parencite[see][chap.~2, sec.~4)]{MakkaiReyes1977}. In order to get the \textit{internal theory} \( \mathbf{T}_{\cat{C}} \) of \( \cat{C} \) in its canonical language, \( \cat{C}\) has to have more structure than just finite limits.  It has to be at least a \textit{regular} category, which we will define shortly. In this case, it is possible to give a list of \textit{regular} axioms \( \Sigma_{\cat{C}} \), that is a set of regular formulas, and prove that \( \mathbf{T}_{\cat{C}} \) is sound in \( \cat{C} \) \parencite[see][chap. 3]{MakkaiReyes1977}. The internal theory \( \mathbf{T}_{\cat{C}} \) is related to \( \cat{C} \) by two expected properties:
%
\begin{enumerate}
\item There is a canonical interpretation \( G \) of \( \mathbf{T}_{\cat{C}} \) in \( \cat{C} \);

\item For any model \( M \) of \( \mathbf{T}_{\cat{C}} \) in a regular category \( \cat{D} \), in any reasonable sense of the term `model', there is a unique regular functor \( I\colon \cat{C} \rightarrow \cat{D} \) such that \( I \) applied to \( G \) is equal to \( M \).
\end{enumerate}

It is of course possible to complete the circle: starting with a regular category \( \cat{C} \), construct its internal theory \( \mathbf{T}_{\cat{C}} \) and then move to its category of concepts \( [\mathbf{T}_{\cat{C}}] \). How are \( \cat{C} \) and \( [\mathbf{T}_{\cat{C}}] \) related? They are in fact \textit{equivalent as categories}, which means that they share the same categorical properties. In other words, as abstract mathematical structures, they are indistinguishable. From this, it is possible to conclude a very important result that every (small) regular category is equivalent to a category of concepts for some theory \( \mathbf{T} \). 

\subsection{The Architecture of Logical Theories}

We now have sketched how a theory in a logical framework can be turned into an instance of an abstract mathematical structure. It should not come as a surprise to learn that a regular theory \( \mathbf{T} \) (resp. a coherent, geometric, etc.) yield a specific kind of abstract category, namely a regular category (resp. a coherent, geometric, etc.). We will fill in some blanks here, for we want to emphasize the existence of kinds of abstract mathematical structures. The existence of these abstract mathematical structures explains why we have introduced these fragments of first-order logic. Logic itself can be organized from the perspective of these structures. We thus get what we call the ``architecture of logical theories'' or the ``architecture of logic''.\footnote{Again, we are being very selective, and our goal is not to be exhaustive. The picture is much more elaborate than what we are presenting here. This is but the tip of the iceberg.}

We will simply state the definition without explaining all the technical details. We refer the reader to the literature. 

A \textit{regular} category \( \cat{C} \) is a category with finite limits\footnote{There are various equivalent definitions of regular categories in the literature. We are following \parencite{Borceux1994}.}, such that 
\begin{enumerate}
	\item Every morphism has a kernel pair;
	
	\item Every kernel pair has a coequalizer;
	
	\item The pullback of a regular epimorphism along any morphism exists and is a regular epimorphism.
\end{enumerate}
This is the purely abstract mathematical structure corresponding to a regular theory \( \mathbf{T}\), but the abstract notion was not abstracted from that construction. It has an independent mathematical existence. The definition does not show automatically how regular logic can be interpreted in a regular category or that a regular theory yields, as its category of concepts, a regular category. But of course, in both cases, it does. 

In a structuralist framework, one has to specify the criterion of identity for the abstract structures given. Thus, we first have to specify what a \emph{regular functor} between regular categories is. Of course, it is a functor that preserves the appropriate structure. In this particular case, a functor $F : \cat{C} \to \cat{D}$ between regular categories is \emph{regular}  if it preserves finite limits and regular epimorphisms. The criterion of identity for regular categories is given by the notion of \emph{equivalence} of regular categories, that is by a pair of regular functors $F : \cat{C} \to \cat{D}$ and $G : \cat{C} \to \cat{D}$ such that $G \circ F \simeq 1_{\cat{C}}$ and $F \circ G \simeq 1_{\cat{D}}$. 

In the same vein, corresponding to coherent theories, we have:

A \textit{coherent} category \( \cat{C} \) is a regular category such that
%
\begin{enumerate}

\item Every subobject meet semilattice \( S(X) \) is a lattice;

\item Each \( f^{\ast}\colon S(Y) \rightarrow S(X) \) is a lattice homomorphism.
\end{enumerate}
A coherent functor between coherent categories is a functor preserving the coherent structure, and the criterion of identity for coherent categories is extracted from that context. 

And, of course, we can add more structure and properties to get other abstract mathematical structures. Let us simply mention a few more abstract structures that are directly related to logic.

A \textit{Heyting category} \( \cat{C} \) is a coherent category in which each \( f^{\ast}\colon S(Y) \rightarrow S(X) \) has a right adjoint, denoted by \( \forall_f \). The last condition is sufficient to entail that each \( S(X) \) is a Heyting algebra, that \( f \) is a homomorphism of Heyting 
algebras and that the right adjoint is also stable under substitution. Heyting categories are common: for any small category \( \cat{P} \), the functor category \( \stcat{Set}^{\cat{P}} \) is a Heyting category. They correspond to theories in intuitionistic predicate logic. Heyting functors are defined in the expected manner.

A \textit{Boolean category} \( \cat{C} \) is a coherent category such that every \( S(X) \) is a Boolean algebra, i.e., every subobject has a complement. Boolean functors between Boolean categories are functors preserving the Boolean structure.

A \textit{pretopos} \( \cat{C} \) is a coherent category having (1) quotients of equivalence relations and (2) finite disjoint sums. Pretopos functors can be defined.\footnote{As we have already mentioned, for many important results, it is enough to consider weaker functors between some of the categories involved, e.g. coherent functors, for they preserve, in these particular contexts, the additional structure.}
The notion of pretopos occupies a central place in the picture, since many of the important theorems about logic can be hooked to that notion. Finally, we have to mention at this stage the notion of Grothendieck topos, which surprisingly sits right at the center of the development of first-order logic and its many variants.\footnote{As already pointed out by Makkai \& Reyes in \parencite*{MakkaiReyes1977}, Giraud's theorem can be interpreted as giving a \emph{logical characterization} of the notion of a Grothendieck topos \parencite[see][chapter~1, section~4]{MakkaiReyes1977}.} There is no need to go on for our purposes. 

Let us immediately point out that the category \( \stcat{Set} \) is regular, coherent, Heyting, Boolean, a pretopos and a Grothendieck topos. And it is even more than just those.

There is an interesting and immediate application of the above constructions. In a classical logical framework, the notion of a interpretation or translation of one category into another one is delicate and complicated. Once we move from a theory \( \mathbf{T} \) to its category of concepts \( [\mathbf{T}] \), there is a very simple and direct way to define it. Indeed, a structure-preserving functor \( I\colon [\mathbf{T}] \rightarrow [\mathbf{T}^{\prime}] \) between (small) categories of concepts is called an \textit{interpretation} of \([ \mathbf{T}] \) in \( [\mathbf{T}^{\prime}] \). (When \( [\mathbf{T}] \) and \( [\mathbf{T}^{\prime}] \) have been constructed from theories, one can verify that it is a legitimate notion of interpretation \parencite[see][chap.~7, p.196]{MakkaiReyes1977}. 


\subsection{The models of a theory as a category}

We now have to consider how a theory \( \mathbf{T} \) can be interpreted in a category \( \cat{C} \). As it is often the case when facing such a situation, the easiest solution is to translate what one does in sets, but express it in the language of the category of sets and finally  move to an arbitrary category with the adequate structure and properties. It can indeed be done  and what we get is a genuine generalization of Tarski's notion of satisfaction  or model. The classical notions of  satisfaction, interpretation, model and truth transfer directly to this new context. However, instead of presenting the nuts and bolts of these definitions, we will jump immediately to the next step.

Since we have constructed \( [\mathbf{T}] \) from \( \mathbf{T} \), and since \( [\mathbf{T}] \) is a category, we can  look directly at the interpretations of the latter. We will illustrate the situation with a coherent theory \( \mathbf{T} \), but starting with the (small) coherent category \( [\mathbf{T}] \) constructed from it. A coherent functor \( M\colon [\mathbf{T}] \rightarrow \stcat{Set} \) is called a (set-)\textit{model} of \( [\mathbf{T}] \). Since \( \stcat{Set} \) is a coherent category, this makes sense. 

It is natural to consider to category of all such models, that is the functor category \( \mdl([\mathbf{T}], \stcat{Set}) \), the \textit{category of all (set-)models} of \( [\mathbf{T}] \). The objects of this category are the models of \( [\mathbf{T}] \), that is coherent functors \( M\colon [\mathbf{T}] \rightarrow \stcat{Set} \), and the morphisms are the natural transformations between models \( \eta\colon M_1 \rightarrow M_2 \). These are the \textit{homomorphisms} of models of \( [\mathbf{T}] \) and they are the traditional model-theoretic structure-preserving functions between models. More generally, for any coherent category \( \cat{C} \), the category \( \mdl([\mathbf{T}], \cat{C}) \) of models of \( [\mathbf{T}] \) in \( \cat{C} \) is defined in the same way. We therefore have  a flexibility that was not available previously.

The category \( \mdl([\mathbf{T}], \stcat{Set}) \) of models of \( [\mathbf{T}] \) in \( \stcat{Set} \) \emph{is}  certainly an instance of an abstract mathematical structure.\footnote{This is also true when we take category different from \( \stcat{Set} \). But then, the structure of the resulting category depends directly on the structure of \( \cat{C} \).} It has, in fact, a lot of structure. It is, among other things, a Grothendieck topos, an important type of abstract mathematical structure.

\subsection{A theory and its models: moving up the ladder}

We have identified some of the abstract mathematical structures that arise from the traditional logical notions. We now have, on the one hand, abstract mathematical structures corresponding to what Lawvere referred to as the ``formal'', and, on the other hand, abstract mathematical structures corresponding to what Lawvere referred to as the ``conceptual''. Of course, these have to be connected and these connections constitute the core of classical logic. 

These connections are themselves part of an abstract mathematical structure. Since \( [\mathbf{T}] \) is a category---the formal side of mathematics---and  \( \mdl([\mathbf{T}], \stcat{Set}) \) is a category---the conceptual side of mathematics---, we can investigate the functors between them. But there is more. There are also functors between theories \( [\mathbf{T}] \to [\mathbf{T'}] \), functors between set-models of theories 
\[ \mdl([\mathbf{T}], \stcat{Set}) \to \mdl([\mathbf{T'}], \stcat{Set}), \] 
functors between models of theories in different categories 
\[ \mdl([\mathbf{T}], \cat{C}) \to \mdl([\mathbf{T}], \cat{D}), \]
 functors between all those and natural transformations between some of these functors!\footnote{We are not being careful here. Some of these are covariant functors, while others are contravariant. We simply want to point at the possibilities at this juncture. We are not developing the theory as such. Again, these details do not affect our main point.}
 In fact, we are in a $2$-category, which is a \emph{genuinely new structure}.  A $2$-category is not merely a category with additional data. Thus, once again, we are in a realm of abstract mathematical structures and many of the results we are interested in will be consequences of this abstract mathematical structure \emph{together with} some specific properties inherent in the situation we are dealing with.

Here are some questions that can now be investigated. The main point here is that these questions make perfect sense, they are entirely natural, whereas it is hard to imagine how they could have arisen outside this mathematical context.

 
%
\begin{enumerate}

\item Given an interpretation \( I\colon [\mathbf{T}] \rightarrow [\mathbf{T}^{\prime}] \) between theories, one can transfer models of \( \mathbf{T}^{\prime} \) to models of \( \mathbf{T} \) by composing with \(I \), that is given a model \( M\colon \mathbf{T}^{\prime} \rightarrow \stcat{Set} \), we get by composition with \( I \) a model \( M \circ I\colon \mathbf{T} \rightarrow \stcat{Set} \). Hence, there is a functor \( I^{\ast}\colon \mdl(\mathbf{T}^{\prime}, \stcat{Set}) \rightarrow \mdl(\mathbf{T}, \stcat{Set}) \). The natural questions to ask pertain to the relations between \( I^{\ast} \) and \( I \). More specifically, are there properties of  \( I^{\ast} \) that imply properties of \( I \)? In particular, is it possible that \( I^{\ast} \) being an equivalence of categories imply that \( I \) is? In words, what are the properties of the conceptual that affect the properties of the formal? 

\item Given a functor \( F\colon \cat{C} \rightarrow \cat{D} \) of the right type (that preserves the right kind of structure in each case), we get a functor \( F^{\ast}\colon \mdl([\mathbf{T}], \cat{C}) \rightarrow \mdl([\mathbf{T}], \cat{D}) \) by composing models \( M \) with \( F \). One question here focuses on the categories \( \cat{C}\) and \( \cat{D} \), more specifically on \( \cat{C} \) and the abstract mathematical structure both these categories are instances of. Thus, is there an abstract mathematical type of structure such that all the models of  \( [\mathbf{T}] \) in \( \cat{D} \) arise from models of  \( [\mathbf{T}] \) in   \( \cat{C} \) and functors \( \cat{C} \rightarrow \cat{D} \)?
\end{enumerate}

Other questions can be formulated, but these are not unlike questions that arise in other mathematical domains, thus relating this formulation of logic with comparable frameworks. To be able to identify what is the common abstract core of logic with other mathematical domains and what is specific to logic is one of the gains of the abstract structuralist approach.


\section{Metalogical theorems from an abstract structural standpoint}

From a structuralist standpoint, once the abstract mathematical structures have been identified, one hopes to be able to prove standard theorems from that vantage. And, indeed, one can. One of the epistemic gains expected from these theorems is the identification of the abstract components involved in various proofs and thus see what is the core structural component upon which these results are grounded. Another expected benefit is the possibility to get genuinely new results which were impossible to get in the classical framework, even impossible to formulate adequately.

\subsection{Completeness and conceptual completeness}

Let us start with what can be considered the pillar of logic in general, namely completeness results. Completeness results for various propositional logics are equivalent to representation theorems for various algebras, e.g., in the case of classical propositional logic, the completeness theorem is equivalent to Stone's representation theorem for Boolean algebras. As we have already mentioned, that most natural context to prove this result is already the context of the category of Boolean algebras and the theorem is done up to isomorphism. 

Moving to first-order logic, it is to be expected that the completeness theorems would amount to representation theorems for certain categories, e.g. regular, coherent, pretoposes, Heyting, Boolean, etc. Indeed, the classical (Gödel) completeness theorem is equivalent to a representation theorem for coherent categories, which can be stated thus: for any small coherent category \( C \), there is a (small) set \( I \) and a conservative coherent functor \( F\colon \cat{C} \rightarrow \stcat{Set}^{I} \). A functor \( F\colon \cat{C} \rightarrow \cat{D} \) is said to be \textit{conservative}\index{Functor!Conservative---} if it reflects isomorphisms, i.e., if \( F(f) \) is an isomorphism in \( \cat{D} \), then \( f \) was already an isomorphism in \( \cat{C} \). Needless to say, the key property is precisely that of being conservative. For what it amounts to is the fact that for any diagram in \( \cat{C} \) such that its image under \( F \) in \( \cat{D} \) is a diagram of a universal morphism, then the original diagram was already a diagram of a universal morphism in \( \cat{C} \). 

As we have already mentioned, the category \( \stcat{Set} \) is coherent and so is the functor category \( \stcat{Set}^{I} \). Since the functor \( F\colon \cat{C} \rightarrow \stcat{Set}^{I} \) is conservative, it follows that \( \cat{C} \) shares all the coherent properties of \( \stcat{Set}^{I} \), and in fact of \( \stcat{Set} \). The equivalence between the representation theorem and the completeness theorem\index{Equivalence of completeness and representation theorems} can be established as follows. Assuming the representation theorem, we start with a coherent theory \( \mathbf{T} \) and construct the category of concepts \( [\mathbf{T}] \) of \( \mathbf{T} \), which is a coherent category. Applying the representation theorem to \( [\mathbf{T}] \), we obtain the completeness theorem. To prove the other direction, we assume the completeness theorem and start with a coherent category \( \cat{C} \). Using the internal language of \( \cat{C} \), one constructs as above the coherent theory \( \mathbf{T}_{\cat{C}} \) of \( \cat{C} \). The models of \( \mathbf{T}_{\cat{C}} \) are then constructed so that they are identical with functors \( \cat{C} \rightarrow \stcat{Set} \). The representation theorem then follows from the completeness theorem for \( \mathbf{T}_{\cat{C}} \).

Two important elements have to be added to the picture. First, the representation theorem for coherent categories is but one representation theorem for a whole collection of relevant categories: regular categories, pretoposes, Heyting categories and Boolean categories. Second, these results in fact follow a general pattern. Indeed, the foregoing representation theorem takes a general, purely categorical form, in other words, there is a crucial part that is purely based on the abstract mathematical structures. Given any categories \( \cat{S} \) and \( \cat{C} \), we can always consider the repeated functor category \( \cat{S}^{(\cat{S}^{\cat{C}})} \).\footnote{This is not an unusual construction in mathematics. Think of the double dual of a finite-dimensional vector space, for instance.} In this situation, there is a canonical functor, the evaluation functor
%
\[
e\colon \cat{C} \rightarrow \cat{S}^{(\cat{S}^{\cat{C}})}
\]
for which, given any object \( X \) of \( \cat{C} \), and any functor \( F\colon \cat{C} \rightarrow \cat{S} \), \( e(X)(F) \) is simply \( F(X) \), the evaluation of \( F \) at \( X \). For any subcategory \( \cat{D} \) of \( \cat{S}^{\cat{C}} \), the same functor \( e\colon \cat{C} \rightarrow \cat{S}^{\cat{D}} \) can be defined. It is then possible to show that the representation theorem for coherent categories is equivalent to the claim that the functor \( e\colon \cat{C} \rightarrow \cat{S}^{\mdl(\cat{C})} \) is conservative. The fact that the evaluation functor is coherent holds on purely general grounds. We therefore have a purely categorical description of the representation theorem. Moreover, in the early seventies Joyal\index{Joyal, A.} demonstrated that the functor \( e \) preserves all existing instances of the Heyting structure in \( \cat{C} \). This automatically yields a representation theorem for Heyting categories and, in turn, a canonical completeness theorem for intuitionistic logic. 


The categorical set-up allows is to consider a stronger claim, called the \emph{conceptual completeness}. 
Given a functor  \( I:  \mathbf{T} \to \mathbf{T}^{\prime} \) and an equivalence of categories between \( \mdl(\mathbf{T}^{\prime}, \stcat{Set}) \) and \( \mdl(\mathbf{T}, \stcat{Set}) \), when is it possible to conclude that \( I \) is also an equivalence of categories? From a categorical point of view, the assumption means that the category of models of \( \mathbf{T}^{\prime} \) is indistinguishable from the category of models of \( \mathbf{T} \). We can think of the functor \( I \) as a translation of \( [\mathbf{T}] \) into \( [\mathbf{T}^{\prime}] \), thus as a case when the latter theory can in principle be more expressive than the former. In a sense, the conceptual completeness can be interpreted as saying that adding new concepts to \( \mathbf{T} \) simply does not modify in any essential way what it can express. This means that \( \mathbf{T} \) has some sort of completeness and in this context it makes perfect sense to say that it is \textit{conceptually complete}. Thus, we say that \( \mathbf{T} \) is \textit{conceptually complete} whenever the following is satisfied: if the functor \( I^{\ast}\colon \mdl(\mathbf{T}^{\prime}, \stcat{Set}) \rightarrow \mdl(\mathbf{T}, \stcat{Set}) \) is an equivalence of categories, then the functor \( I\colon \mathbf{T} \rightarrow \mathbf{T}^{\prime} \) was one already. This literally means that by moving to \( \mathbf{T}^{\prime} \), we did not add anything essentially new to \( \mathbf{T} \), although we might have thought we had, and this information was obtained by looking at the \emph{categorical structure of the category of models of the theories}. We can conclude that a certain logical framework, say an equational theory, is enough to characterize a type of structures, from the categorical structure of the category of models. Conceptual completeness is in fact equivalent to a standard result of model theory, namely Beth definability theorem. However, one of the advantages of working in the categorical framework is that categorical methods make it possible to prove results which might not be accessible otherwise, for instance, a constructive proof of this result for intuitionistic logic.\footnote{See \parencite{Pitts1989} for a categorical proof of conceptual completeness of intuitionistic first-order logic.}


It is possible to strengthen the conceptual completeness theorem. In the latter, we assume as given a functor \( I:  \mathbf{T} \to \mathbf{T}^{\prime} \). Is it possible to start with an equivalence of categories \( \mdl(\mathbf{T}, \stcat{Set}) \to \mdl(\mathbf{T}^{\prime}, \stcat{Set}) \) and construct from it an equivalence \(  \mathbf{T} \to \mathbf{T}^{\prime} \)? This is  much stronger theorem, but it can be proved under certain circumstances. It says that a logical theory is completely characterized by the categorical structure of its category of models. In some sense, the conceptual determines the formal, up to equivalence. In the case of propositional logic, it amounts to a form of Stone duality, the latter being formulated entirely \emph{within} the category of Boolean algebras, and not as the existence of an equivalence of categories between the category of Boolean algebras and the category of Stone spaces. The strong conceptual completeness asserts that the Lindenbaum-Tarski algebra of a propositional theory can be recovered from its space of models---the ultrafilters on the given Boolean algebra. A theory for which the theorem can be proved is said to be \textit{strongly conceptually complete}. A different way to formulate this result is to say that if \( \mdl(\mathbf{T}, \stcat{Set}) \) and \( \mdl(\mathbf{T}^{\prime}, \stcat{Set}) \) are equivalent, then \( \mathbf{T} \) and \( \mathbf{T}^{\prime} \) are equivalent too. Whereas conceptual completeness is a local phenomenon, since it depends on the interpretation \( I \), strong conceptual completeness is a global phenomenon, since there is no underlying interpretation at hand. The construction of \( \mathbf{T} \) can be thought of as a case of abstracting certain data out of another, more ``concrete'', situation. Finite limit categories of concepts are strongly conceptually complete\footnote{Thus they are the so-called Barr-exact categories. A Barr-exact category is a regular category in which every equivalence relation is a kernel pair \parencite[see][]{Makkai1990}.}, although the original result applied to (Boolean) pretoposes. If the category of models is adequately enriched in a precise technical sense, then in these circumstances first-order classical logic is strongly conceptually complete \parencites[see][]{Makkai1988}{Makkai1990}[for a a different proof which is build with higher-dimensional categories in mind, see][]{Lurie2019}.  Notice that it is hard to see how this theorem could even be formulated outside the context of category theory.

It is impossible not to mention the fact that strong conceptual completeness theorems are closely related to dualities. In fact, they are equivalent in a precise technical sense to dualities. The only thing we want to underline is that these results are proved in the context of 2-categories. Thus, it is not only that the natural set-up involves 2-categories, but that important theorems require 2-categorical (even bicategorical) concepts. We cannot, in such a short paper, present these in any comprehensible manner. 

\subsection{Syntax and abstract completeness}

From the above considerations, the reader might feel that we have entirely left behind syntactical considerations, more specifically formal deductions. Therefore, it might seem like the categorical completeness results are not quite the same as the classical results which assert that semantical consequences of a theory are provable in a fully specified formal system. This is not the case. For one thing, we have not abandoned the syntax, nor the formal systems in these investigations. But there is an additional point to make, for it brings to the fore a way of dealing with the syntax of theories that emerged naturally from the context of categories, namely the idea of a sketch and its generalizations. 

From Lawvere's thesis, category theorists toyed with the idea that a theory could be presented directly in the form of a category or some graphical variant thereof. In this spirit, sketches were introduced by Charles Ehresmann in the early 1960s and developed afterwards by him and his school \parencite[see][for instance]{Ehresmann1967,Ehresmann1968,Lair2001,Lair2002,Lair2003}.  A sketch, which is a specific kind of (oriented) graph, is a new kind of syntax, specifically tailored to do categorical logic. It gives directly in a graphical way the syntactical and proof theoretical content of a theory.

We will give one definition of the notion of sketch\footnote{See \parencite{BarrWells1990, BarrWells2005} for an introduction to sketches.}. A sketch $\mathcal{S} = (G, D, L, C)$ is given by a graph $G$, a set $D$ of diagrams in $G$, a set $L$ of cones in $G$ and a set $C$ of cocones in $G$. We can consider the category of sketches by stipulating that a morphism of sketches is a homomorphism of graphs which preserves the diagrams, the cones and the cocones. It is easy to see that any category $\mathbb{C}$ has an underlying sketch $\mathcal{S}_{\mathbb{C}}$. A \emph{model} of a sketch $\mathcal{S}$ in a category $\mathbb{C}$ takes all the diagrams of $\mathcal{S}$ to commutative diagrams, all the cones of $\mathcal{S}$ to limits  of $\mathbb{C}$ and all cocones of $\mathcal{S}$ to colimits of $\mathbb{C}$. A morphism of models is a natural transformation. Thus, we can reproduce what we did above with theories, namely we can construct the category of models $\mdl(\mathcal{S}, \mathbb{C})$ of a sketch $\mathcal{S}$ in a category $\mathbb{C}$.

It is natural to consider sketches in which there are no cocones and only discrete and finite cones, or in which there are no cocones and only finite cones, etc. Sketches organized themselves with respect to these natural choices and they correspond to various logical theories. Thus, there are finite product sketches (a FP-sketch), left exact sketches (a LE-sketch), regular sketches, coherent sketches, etc. 

In this framework, it is natural to ask which categories are sketchables: is it possible to characterize categories that are equivalent to categories of models of a type of sketch? There are positive answers to that question and it naturally brings us, when the most general kinds of sketches are considered, to infinitary logic $L_{\infty, \infty}$ \parencite[see][]{Lair1981,Makkai1989}. 

Generalization of the notion of sketch has led Makkai to develop a categorical proof theory and establish a completeness theorem along the classical lines, that is proving that a formula is formally provable in a theory if and only if it is true in all models of the theory. Interestingly enough, the set-up still rests upon the categorical representation theorems, but it is enriched with a categorical notion of formal proof in the set-up of (generalized) sketches. \parencite[see][]{Makkai1997a,Makkai1997b,Makkai1997c}. 

\subsection{Incompleteness}

We  have to say a few words about Gödel's incompleteness theorems. Is it possible to identify abstract mathematical structures that underly these theorems? Is it possible to deduce these theorems from a theorem or theorems about these abstract mathematical structures? There are some pieces in place, although the complete picture---no pun intended---has still to be presented.

First, already in the 1960s, Lawvere presented a categorical analysis of various phenomena related to Gödels's incompleteness theorems. In his \parencite*{Lawvere1969b}, Lawvere presents what he takes to be the abstract mathematical structure underlying Cantor's theorem that there is no surjection $\mathbb{X} \to 2^{\mathbb{X}}$ and its variants in the heads of Russell, Gödel and Tarski. The starting point here is the notion of a cartesian closed category. A cartesian closed category is a nice example of a categorical doctrine since it can be given entirely by stipulating the existence of certain adjoint functors to elementary, that is first-order, functors. More precisely, a cartesian closed category $\mathbb{C}$ is a category such that 
\begin{enumerate}
	\item The functor $! : \mathbb{C} \to 1$ has a right adjoint;
	\item The diagonal functor $\Delta : \mathbb{C} \to \mathbb{C} \times \mathbb{C}$ has a right adjoint, namely the product functor;
	\item For each object $X$ of $\mathbb{C}$, the functor $X \times (-) : \mathbb{C} \to \mathbb{C}$ has a right adjoint $(-)^X$.
\end{enumerate}

Given a cartesian closed category $\mathbb{C}$, it is possible to then define what D. Pavlovic has called a ``paradoxical structure'' on an object of $\mathbb{C}$ that satisfies a fixed-point property. By specifying the adequate cartesian closed category $\mathbb{C}$ and the paradoxical structure on objects of $\mathbb{C}$, it is possible to prove Cantor's theorem, Russell's paradox, Gödel's first incompleteness theorem, Tarski's theorem of the impossibility of defining truth in a theory and many others (see \cite{Pavlovic1992} and \cite{Yanofsky2003}).

In a series of unpublished lectures presented in the 1970s, André Joyal introduced another abstract mathematical structure, in a precise sense weaker than Lawvere's proposal, to pursue the analysis of Gödel's incompleteness results, in particular the second theorem, namely what he called `arithmetical universes'. Very roughly, an arithmetic universe $\mathbb{U}$ is a pretopos such that the free category object constructed from a graph object in $\mathbb{U}$ exists. This is the abstract structure in which one can do recursive arithmetic and prove versions of the two incompleteness theorems (see \cite{Maietti2010} for a different definition).

\subsection{Type theories}

We will be very succinct, not because this area is not important, quite the contrary, but simply because there is no need to cover everything in details given the goal of this paper.  

The first and well-known result in this area is the correspondence between cartesian closed categories with a natural number object and typed $\lambda$-calculus. We finally get to elementary toposes. These were introduced by Lawvere and Tierney in 1970 to provide an elementary treatment of sheaves over a site, thus of Grothendieck toposes. It is another remarkable example of a categorical doctrine.
An \emph{elementary topos} $\mathcal{E}$ is a category with finite limits, cartesian closed, and has a subobject classifier. These three conditions do amount to the existence of certain adjoint functors to given (elementary) functors. As is well known now, it is possible to construct an intuitionistic type theory from a given topos $\mathcal{E}$ and, conversely, it is possible to specify an intuitionistic type theory such that its conceptual category is an elementary topos and it can be interpreted in an elementary topos. There is then a correspondance between categorical properties of the topos and logical properties of the type theory \parencite[see][]{Boileau1981,Lambek1988}.

The same can be said about homotopy type theory. Homotopy type theory comes form Martin-Löf's intensional type theory \parencite[see][]{HoTT2013}. It has models in various categories, but a homotopy type theory ought to correspond to a kind of abstract categories. It has been conjectured, by Steve Awodey, that homotopy type theory should correspond to the internal logic of higher-dimensional elementary toposes. As of this writing, the full conjecture has still to be proved, although certain advances have been made \parencite[see][]{Kapulkin2018}. 

\subsection{One last thing...}

Last but not least, connections between linear logic and category theory appeared almost immediately after the creation of linear logic by Girard in \parencite*{Girard1987} \parencite[see][]{Lafont1988,Seely1989}. It took almost twenty years of research before a consensus emerged as to what constitutes a categorical model of linear logic  \parencite[see][]{Bierman1995,BluteScott2004,Mellies09,DePaiva2014}.  We will not introduce nor discuss the categorical framework here. It would require defining and explaning a lot of categorical structures, e.g. symmetric monoidal categories, symmetric monoidal adjunctions, etc., as well as an explanation of  how the various frameworks proposed converge towards a basic structure. The point is: we may be seeing the beginning of a stable picture that will allow us to start building a conceptual interpretation of linear logic. In as much as homotopy type theory seems to be intimately connected to the basic constituents of spaces, the ``atoms of space'' to use Baues's expression in \parencite*{Baues2002}, namely homotopy types, linear logic seems to be intimately tied to generalized vector spaces and the mathematics inherent to the latter \parencite[see][]{Mellies2020}. If this reading is correct, it may lead to new interpretations and developments of conceptual spaces and the categorical structures would naturally find their place in that context. But this is sheer speculation at this point and it does not affect our main point.

\section{Conclusion}

We  insist that this way of framing logic and metalogic is a direct continuation of mathematical logic as it developed in the first half of the 20\textsuperscript{th} century and the rise of the abstract axiomatic method at the same time. Category theory itself is an offspring of this period, and as such, does not constitute a radical methodological change. It is, undoubtedly, a rise in abstraction. It reveals new types of abstract mathematical structures. 

We hope we have convinced the reader that it is possible to identify the abstract mathematical structures underlying (fragments of, and extensions of) first-order logic and type theories. It is also possible to see how the important metalogical results correspond to theorems on these abstract mathematical structures. Finally, the invariance property at the core of any abstract mathematical structuralism comes naturally and automatically in this framework. Thus, the standard logical systems---and some non-standard logical systems as well---find a natural place in this structuralist context. Pure logic is seen as a specific type of abstract structure. We can thus answer our own challenge positively and precisely. We leave to other structuralists to provide their answer to our challenge.



\end{artengenv}