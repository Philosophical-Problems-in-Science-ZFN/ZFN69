\begin{recengenv}{Roman Krzanowski}
	{Contemporary Polish Ontology. Where it is and where it is going}
	{Contemporary Polish Ontology. Where it is and where it is going}
	{\textit{Contemporary Polish Ontology}. Skowron, B. (ed.), Philosophical Analysis, 82. Berlin; Boston: De Gruyter, 2020. pp.320.}

The \textit{Contemporary Polish Ontology} is a~collection of recent works by Polish ontologists. The volume includes thirteen papers and introduction as well as discussion sections. The authors published in the volume are associated with the International Center for Formal Ontology (ICFO) at the Faculty of Administration and Social Sciences, Warsaw University of Technology, and they are affiliated with several research centers across Poland. The articles in the book therefore reflect, to some degree, the state of current ontological research in Poland.

\enlargethispage{1.5\baselineskip}
The volume is not intended to be read from cover to cover, as it includes a~diverse collection of topics united only by a~common ontological vantage point. Few people, I~suspect, would be interested in reading each chapter. So, how should we approach this book, and why might someone want to read it? It seems there may be two reasons for doing this and two ways to do it. For example, someone may want a~primer on the state of ontological research in Poland, such as who is doing what, what topics are being investigated, and what questions are being posed. Alternatively, someone may be interested in a~specific topic and what has been done in that area. If you fall into the first group, you should read the introduction written by Bartłomiej Skowron and the final chapter, \textit{An Assessment of Contemporary Polish Ontology}
%\label{ref:RNDfUPgKvXsSa}(Skowron, 2020, pp.271–294),
\parencite[][pp.271–294]{skowron_contemporary_2020}, %
 which has been coauthored by several researchers, but particularly the section written by Skowron. This section provides a~succinct yet comprehensive review of the state of Polish ontology, together with some added historical background. If you want to then go further, you may find specific topics that pique your interest by reading the introduction before diving deeper into the volume. If you fall into the second group, namely being interested in just a~specific topic, or are concerned with specific ontological questions, you should start with the introduction and then establish which chapters are of most interest to you.

So, what topics are presented in the volume? Tomasz Bigaj discusses the concept of symmetry in structures, proposing its redefinition and indicating its possible impact on Quantum Theory. Mariusz Grygianiec, meanwhile, analyzes the possibility of integrating the concept of personal identity into animalism and that of the Simple View. Next, Filip Kobiela analyzes the concept of the present and Ingarden's take on this, thus formulating the “outline of the ontological theory of relativity of the duration of the present.'' Zbigniew Król and Józef Lubacz then take over by exploring the epistemic conditions for the knowledge of existence, suggesting how this type of knowledge may be obtained in the conscious subject. Andrzej Biłat then attempts to formulate the classical conception of philosophy, which can be understood as the philosophy of Plato and Aristotle, and shows its compatibility with contemporary logic and science. Urszula Wybraniec-Skardowska, meanwhile, tries to formulate the ontology of language, where language has a~particular ontological status. Krzysztof Śleziński explores Bornstein's concepts of general ontology and shows how his ideas live on in the current research into spatial logic. Next, Janusz Kaczmarek explores the concepts of topological ontology and their relation to Leibniz's \textit{Monadology}. Coming back to Krzysztof Wójtowicz, he discusses the necessary conditions for the existence of mathematical objects and demonstrates the import of this discussion to the realism–antirealism debate in the philosophy of mathematics. Rafał Urbaniak, meanwhile, reviews approaches to the formulation of mathematical theories, focusing on the status of neologicism. Jacek Paśniczek explores the application of an algebraic framework based on the De Morgan lattice for the representation of the ontological features of possible worlds and situations. Next, Marek Magdziak presents a~logical study on the concepts pertaining to the notion of action, while Michał Głowala endeavors to demonstrate how some conceptual tools of scholastic philosophy can be helpful in resolving the current debates about the ontologies of intentionality and powers.

\enlargethispage{-.5\baselineskip}
All in all, the \textit{Contemporary Polish Ontology} is certainly worthy of attention. It provides a~panorama of the various ontological topics that are currently being pursued in Poland. Anyone with an interest in ontological questions will likely find an engaging paper. For those interested in the state of Polish philosophy, meanwhile, the discussion in the final chapter, \textit{An Assessment of Contemporary Polish Ontology}, is worthy of particular attention. This discussion—which has been coauthored by Bartłomiej Skowron, Tomasz Bigaj, Arkadiusz Chrudzimski, Michał Głowala, Zbigniew Król, Marek Kuś, Józef Lubacz, and Rafał Urbaniak—lists several challenges that are being faced by Polish ontology and philosophy in general. The presented facts are rather well known but rarely explicated, and the remedy for these shortcomings is less obvious and seemingly out of reach for now. Some of the contributing authors to this chapter question whether we are doing good research (which we seem to be), why are we ranking so low in the world market for ontological ideas, and why is what we do is rather unknown (albeit with notable exceptions) outside the limited circle of Polish universities. Our recent history and the relative obscurity of the Polish language clearly offer some excuses here, but these do not entirely explain everything away. The final section of this chapter, Skowron's excellent essay, takes a~more positive note and is essential to retaining a~balanced, yet critical, perspective on Polish Ontology. Unfortunately, the essay is rather short.

All the papers in the volume are clearly written and well organized, but a~few additions could have improved the value of the collection. For example, the papers could have placed the presented research within the context of similar discussions outside Poland. In other words, the authors could have provided a~well-documented background to the topics. If Polish philosophy is to come out into open and avoid the potential accusations of navel-gazing, it should do so by relating its findings to the dialogue among the international ontological community about current problems. Naturally, this should be done while preserving our unique and original perspective, although I~concede this may be a~challenge. With the notable exception of a~few papers, this larger context is not really emphasized in the \textit{Contemporary Polish Ontology}.\footnote{To gain a~wider perspective on ontology, we may look for comparison at the selection of topics presented during the latest Joint Ontology Workshops' (JOWO) meeting and see where \textit{Contemporary Polish Ontology} fits in. The program of the JOWO 2020 Episode VI: \textit{The Bolzano Summer of Knowledge} is available at {\textless}\url{https://www.iaoa.org/jowo/2020}/{\textgreater}.}

What is also a~little concerning is the limited references that are supplied in some articles, again with the notable exception of several papers. There is also a~notable absence of some hotly pursued topics in current ontology, such as the ontologies of computing objects, sciences (e.g., biology, genetics, medicine, engineering), and system design, as well as object ontology. What is more, as we have indicated, what would potentially improve the book's reception is a~short chapter that would place the book (assuming that the intention is to give an overview of Polish ontology) in the context of ontological research outside Poland. This information, as we have said, maybe found in the introduction to several essays, but unfortunately it is dispersed and not very detailed and systematically exposed. Moreover, the collection would provide a~more rounded image of %
contemporary Polish ontology if it could avoid the potential critique of assuming a~limited perspective, such as if the volume mentioned work originating from other places not associated with the ICFO. (In his review, Skowron does provide a~more comprehensive and inclusive list of Polish philosophers dealing with ontological themes.)\footnote{For example, the volume could mention the work of Michal Heller on existence in physics
%\label{ref:RND2KYXuSXp65}(Heller, 2018)
\parencite[][]{heller_what_2018} %
 or Edward Malec's work on the existence of black holes 
%\label{ref:RND4urmig3mQw}(Malec, 2018).
\parencite[][]{malec_black_2018}. %
 The volume from which these two works are cited is entitled \textit{On what exists in physics}.}

\enlargethispage{-.5\baselineskip}
To summarize this review, we may say that \textit{Contemporary Polish Ontology} does serve several detailed ontological studies from leading Polish research centers. The range of topics is rather broad, and anyone interested in ontology will certainly find something of interest. In hindsight, however, one may question why certain topics that are prevalent in the current ontological debates are absent from the volume. Maybe the selection of topics reflects the specificity of the Polish school, however. The clearly written introduction guides the reader to specific topics and therefore removes the need to read all of the abstracts. The final chapter, in contrast, is very much addressed to the Polish audience: One may call it a~manifesto of sorts, a~“What to do?'' in the world of Polish ontology and Polish philosophy in general. From the book, one may come to know the key Polish researchers (their email addresses are provided) working in ontology. Such information is valuable, and the volume also serves the secondary purpose of being a~“Who's who'' in Polish ontology.

\autorrec{Roman M. Krzanowski}


\end{recengenv}