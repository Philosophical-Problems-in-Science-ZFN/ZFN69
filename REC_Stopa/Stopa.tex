\begin{recengenv}{Mariusz Stopa}
	{Category Theory in the hands of physicists, mathematicians, and philosophers}
	{Category Theory in the hands of physicists, mathematicians, and philosophers}
	{\textit{Category Theory in Physics, Mathematics, and Philosophy}, Kuś M., Skowron B. (eds.), Springer Proc. Phys. 235, 2019, pp.xii+134.}

The book under review is a result of the conference under the same title which was held in Warsaw, Poland in November 2017. As far as I know, it was the first conference in Poland's history on the subject of cat\-e\-go\-ry theory (CT) and its applications. Warsaw is an excellent place for such an event if only because of its traditions of the Lvov–Warsaw School, but also being the place where Samuel Eilenberg (one of the founding fathers of CT) was born and defended his Ph.D. thesis. The majority of the organizers and speakers were Polish. Nevertheless the significance and impact of this conference, mainly thanks to the proceedings, is definitely international. The present volume of ZFN, which itself is devoted to the conference proceedings of the next great event of this kind, shows that the tradition of Polish conferences on the subject of CT and its applications emerges. It could only be wished it would thrive in the future.

The publication consists of 10 chapters, which are all separate articles, the majority of which are papers presented at the conference. The subjects of the papers are very diverse, which itself evinces how vast are the applications of CT to different areas of knowledge.\footnote{The reader interested in the subject of broad applications of CT may benefit also from e.g. the great book \parencite{awodey_category_2010} (ZFN published its Polish review by Michael Heller \parencite*{heller_filozoficznie_2018}).} The reader not familiar with at least the basics of CT will be much hindered in the study of them.\footnote{Textbooks in CT that may be consulted are e.g. \parencite{awodey_category_2010, smith_category_2018, mclarty_elementary_1995}.} However, not all the papers hinge on the technicalities of CT (see the first four chapters). Others also require some familiarity with further knowledge (mainly from physics for chapters 7--9). 

The book begins with a few pages of introduction, which presents the work's context and succinctly discusses each chapter. The first article, \textit{Why Cat\-e\-gories?}, written by M. Kuś, B. Skowron, and K. Wójtowicz, is also a kind of introduction to the whole book, but from a different perspective. It poses a question (and attempts to answer it, at least partially) about reasons for such a significant increase in popularity of CT and its applications in recent years. The great diversity of the next chapters in the book is a good, though still only partial, example of this phenomenon. This paper presents some aspects of the origins of CT, its relations with, and possible applications to philosophy and physics. Within applications to philosophy, it discusses structuralism in the philosophy of mathematics, foundations of mathematics, unity of mathematics, and metaphysics. In connection with physics, it examines mainly quantum mechanics and its ontological foundations. Nonetheless, the text is more than a review. The authors, when discussing the above mentioned different applications of CT, also give examples of their main claim: ``cat\-e\-go\-ry theory is a formal ontology that captures the relational aspects of the given domain in question'' (p.~1).\footnote{References with no information except the page number refer to the book under review.} These examples show how the formal-ontological shift brought by CT, meant as the shift from the ``standard and natural attitude towards the objects, as if they were individual subjects of properties [\ldots] [to] a form of pure relationality'' (p.~18), sheds some new light on the old problems.


The next chapter, \textit{Category Theory and Philosophy}, by Z. Król, deals with relations between CT and philosophy. One of the topics addressed in this article, although not the main one, is how set theory (ST) and CT are similar and different. The author considers them more broadly than just the formal theories: ``CT and ST are not singular formal theories, but rather open domains accompanied by the relevant methods and styles of consideration, together with some basic concepts which can be investigated within many different formal theories'' (p.~22). The main part of this article is a case study of certain classical problems in philosophy and the question of the usefulness of CT in dealing with them. The study is rather general without deeper analysis, albeit the author gives some details and references to literature. It starts with remarks on the great influence of mathematics---in general---upon philosophy in its history. The studied cases concern some issues in ontology (such as monism vs pluralism, or platonistic philosophy of mathematics) and epistemology.

K. Wójtowicz in his paper, \textit{Are There Category-Theoretical Explanations of Physical Phenomena?} (chapter 3), addresses a question of mathematical explanation in science. He asks especially whether CT may give explanations of physical phenomena. To answer this question he first analyzes the general problem of the explanatory role of mathematics in physics, assuming (as a working hypothesis) that ``there are genuine mathematical explanations in science'' (p.~37). He admits that ``it cannot be denied that CT contributes to our understanding of physics'' (p.~33) and gives some illustrations for this, mainly within the Topos Quantum Theory. However, after noting first that ``if a model has no predictive potential at all, it is doubtful whether it really has explanatory character'' (p.~38), his central claim is that ``in this sense there are no cat\-e\-go\-ry-theoretical explanations of physical phenomena, in spite of there being mathematical explanations'' (p.~38). The author considers the contribution of CT to be ``on the metatheoretical rather than the theoretical level'' (p.~40). He eventually writes: ``CT offers `meta-abstract explanations'$\,\!$'' (p.~42). I have the feeling that this article underestimates rather than overestimates even the hitherto relevance of CT to other disciplines. Besides, maybe the future will show more.


Chapter 4, entitled \textit{The Application of Category Theory to Epistemic and Poietic Processes}, by J. Lubacz, tries to explore the possibility of applying CT and its notional framework to ``the analysis and monitoring of progress in the unfolding of [\ldots] processes,'' (p.~45) such as acquiring knowledge and some activity that results
in the creation of artefacts. These are called epistemic and poietic processes, respectively. The style of this paper is definitely philosophical. After the introduction, the author presents some considerations about the processes in question in the broader context of philosophy as well as presenting his own proposition of the possible ``conceptual structure of the pattern and dynamicity of epistemic and poietic processes'' (p.~48). Next, he ponders the potential application of CT to epistemic and poietic processes. He rightly notes that ``it must be clearly stated that the apparatus of CT can only be employed for those conceptual components of epistemic and poietic processes which are expressible in some formal language and form'' (p.~50). He then develops his own proposal of the possible use of CT in this context, which is, however, rather general and preliminary in nature.

\enlargethispage{-.5\baselineskip}
In his paper, \textit{Asymmetry of Cantorian Mathematics from a Categorial Standpoint: Is It Related to the Direction of Time?} (chapter 5), Z. Semadeni addresses an interesting feature of the \textit{Cantorian Mathematics}. In his approach, \textit{Cantorian Mathematics} is referring to ``basic mathematical structures of algebra, topology, functional analysis etc. expressed in terms of set theory, as they were conceived prior to the emergence of cat\-e\-go\-ry theory, i.e., by the middle of the 20\textsuperscript{th} century'' (p.~55f). The author notes and explains briefly that CT, in a certain sense (with respect to the reversal of arrows), is symmetric. However, when we consider some cat\-e\-gories with objects being structures taken from \textit{Cantorian Mathematics}, then the products and coproducts of these objects (separately in each cat\-e\-go\-ry) each have a different 'style.' Namely, while almost all examples considered (and the author gives quite a few of them) of products turn out to be ``the cartesian product endowed with a suitable structure'' (p.~57) (and the remaining two examples can be in some way cured or revised alike), coproducts, however, ``may be markedly different from each other'' (p.~57). For some cat\-e\-gories coproducts ``are based on the same construction, namely on the disjoint union'' (p.~57), but ``in other cat\-e\-gories coproducts may differ basically'' (p.~57). As an example, one of many the author gives, let me note the cat\-e\-go\-ry \textbf{Grp} (of groups and their homomorphisms) in which the coproduct is the free product of groups, whereas in its full subcat\-e\-go\-ry of abelian groups the coproduct is the (external) direct sum of groups. The author notes that similar asymmetry concerns also equalizers and coequalizers (``albeit in a much milder form'' (p.~59)), and other limits and colimits. At this point, he poses a philosophical question: ``What features of \textit{Cantorian Mathematics} lie behind this asymmetry?'' (p.~60), and suggests that it follows ``from the asymmetry of many-to-one relationship in the notion of a function $ f:X\to Y $'' (p.~60). In the last two paragraphs, we find an interesting discussion showing that the asymmetry between products and coproducts changes, so to speak, direction, when instead of considering examples of an ``algebraic'' nature one considers structures of ``coalgebraic'' nature connected with one-to-many relation (``mostly stimulated by Computer Science'' (p.~61)).

\enlargethispage{-.5\baselineskip}
The next article, entitled \textit{Extending List’s Levels}, by N. Dewar, S.C. Fletcher, and L. Hudetz is an interesting extension and modification of the unified framework for modeling different types of levels (descriptive, explanatory and ontological) proposed by C. \textcite{list_levels_2019}. The paper is well written---particular notions are clearly introduced and commented on and occasionally important and helpful examples are given. First, the authors succinctly review List's approach and correct a minor defect. Next, they analyze the relationship between supervenience and reduction in this setting. In general, supervenience does not entail reduction (and the reader is familiarized through a simple example), but the authors have shown that if the levels and supervenience maps fulfill certain additional conditions (they have to be compatible and jointly characterizable, the notions being defined and commented in the text), then supervenience does entail reduction. They note, moreover, that ``in many cases of supervenience between scientific levels of description, this [fulfilling the above-mentioned condition(s)] can be expected. So it is quite plausible that in many cases of interest, supervenience and reduction of levels go hand in hand'' (p.~79). After these considerations, the authors propose two extensions of List's framework: from supervenience maps treated as (total) functions to partial maps, and from surjective maps to non-surjective ones. These generalizations open new possibilities, described in the text. Subsequently, the authors move on to the most important, in my opinion, part of their work, namely to the modification of List's framework which involves considering levels not only as elements of a certain poset (or objects of a posetal cat\-e\-go\-ry) but as cat\-e\-gories of structures (more precisely, they suggest ``to represent a level of description, $ L $, as a pair $ \left\langle \mathbf{L},\Omega\right\rangle  $ consisting of a description language, $ \mathbf{L} $, and a cat\-e\-go\-ry, $ \Omega $, of $\mathbf{L}$-structures'' (p.~73)). This means that in order to specify a level of description ``one does not only specify its structures but also the morphisms (admissible transformations) between these structures'' (p.~73). The choice of morphisms ``reflects which expressions of $ \mathbf{L} $ are taken to be meaningful within the level $ L $'' (p.~73). Moreover, in this setting supervenience relations between the levels are viewed as functors. Another extension of the original framework is made by allowing ``all sorts of functors to be included in a system of levels of description'' (p.~77), rather than just the supervenience functors. The authors give also some examples involving these generalizations and note that the extended framework better serves for a philosophy of science in general. This last generalization (of treating levels as cat\-e\-gories) is far-reaching, as ``taking levels as cat\-e\-gories themselves demands a more robust use of cat\-e\-go\-rial ideas [such as natural transformations, adjunctions, and others] that could also prove to be more fruitful'' (p.~79).

\enlargethispage{-.5\baselineskip}
The next three chapters deal with applications of CT to physics. The first of them, written by K. Bielas and J. Król, titled \textit{From Quantum-Mechanical Lattice of Projections to Smooth Structure of $ \mathbb{R}^4 $}, uses CT to relate quantum algebra structure with the smooth structure of spacetime. The paper is a work in progress and is connected with some earlier article by the authors and another scholar, namely \textcite{krol-2017}. After the introduction, in which the authors already bring in some key concepts, the next section outlines some quantum mechanical preliminaries, i.a. introducing a complete orthomodular lattice of projections on a Hilbert space associated with the initial quantum system. This lattice (denoted as $ \mathbb{L} $) is an algebraic basis of the so-called quantum logic. As is well known, generally (whenever $\text{dim}\,\mathscr{H}>2$) $ \mathbb{L} $ is not Boolean, which means that logic of a quantum system defined in this way is not classical. In order to get the connection with the classical world the authors, in the next section, look at the subalgebras of $ \mathbb{L} $ which are Boolean. Each such Boolean subalgebra, they argue, is in a certain way ``to be considered as a local, classical frame of reference for a quantum system'' (p.~87). Subsequently, they show a way to construct an orthomodular lattice from its Boolean subalgebras as a suitable colimit in the cat\-e\-go\-ry of so-called partial Boolean algebras and appropriate homomorphisms (one has to do it in this larger cat\-e\-go\-ry, which extends the cat\-e\-go\-ry of orthomodular lattices and lattice homomorphisms). Then it is shown, in the next section, how one can arrive at the smooth manifold by means of some cat\-e\-gor\-i\-cal constructions, namely the authors conclude that ``given a smooth manifold, it is always the colimit of its atlas'' (p.~89).\footnote{The reader might find it helpful to note that there is a certain confusion with the notation: on p.~88, $ V_i $ is a subset of $ \mathbb{R}^n $, whereas in the Corollary 2, on p.~89, $ V_i $ is a subset of the manifold. It would be more natural to denote on p.~89, in accordance with the notation from p.~88, $ V_i $ as $ U_i $, and $ W_i $ as $ V_i $.} Finally, the authors try to relate the quantum structure with the differential (smooth) structure of the manifold. However, it is still an open problem if certain correspondence has a functorial character. In the \textit{Discussion} section some further considerations are addressed, i.a. the cardinality of the smooth atlas of exotic $ R^4 $. 


Differential structure of a manifold, or rather its enrichment in terms of a so-called formal manifold is a key tool of the next paper, \textit{Beyond the Space-Time Boundary}, by M. Heller and J. Król. The article is quite technical, preparatory for further research, but also fundamental and opening a truly intriguing path for new explorations.\footnote{The interested reader may consult also other papers in this subject by these authors, e.g. see \parencite{heller-krol-2016, heller-krol-2017}.} In order to ``cross the boundary'' of a singular spacetime, they use the so-called Synthetic Differential Geometry (SDG), which is a cat\-e\-go\-rical version of standard differential geometry and is based on intuitionistic logic. After the introduction, the authors offer many new notions (known in the literature) needed for further considerations and their model in the next sections. Among others, one considers various kinds of infinitesimals, which enrich the standard structure of $ \mathbb{R} $ (or a manifold in general) and make differentiation a purely algebraic operation. The authors give a nice image: ``We may imagine that they [infinitesimals] constitute the entire world inside every point of $ \mathbb{R} $, a sort of a fiber over $ x\in\mathbb{R} $'' (p.~96). The reader not familiar with such infinitesimals may find the definition such as ``$ D=\{x\in R | x^2=0\} $'' (p.~97) astonishing. Let me only note that the intuitionistic logic plays one of the key roles here. In SDG such an object $ D $ obviously does not reduce to $ \{0\} $. It comprises so-called nilsquare infinitesimals, which are different, but at the same time indistinguishable, from $ 0 $.\footnote{In the intuitionistic logic, double negation does not, in general, imply identity (double negation elimination is not a theorem), so not being different from $ 0 $ (not being not equal $ 0 $) does not imply being identical to zero or, in other words, indistinguishability does not, in general, imply identity.} The whole collection of (countably many) different kinds of infinitesimals serves to distinguish various kinds of neighbourhoods, which in turn allow us to define various kinds of the so-called monads. In the proposed model, for example when thinking about the evolution of the universe back in time, these monads, which are ordered (relative to each other) and represent various kinds of differentiability properties, provide suitable structures to describe the evolution after crossing the singular boundary (which from the physical point of view might possibly be identified with Planck's threshold). The authors note that the model ``does not pretend to describe the actual evolution of the universe. At its present stage of development, it is nothing more than a toy model'' (p.~96). The toy models, however, can play a surprisingly significant role in the development of more physical ones! Two appendices give more details, for example about the way infinitesimal spaces may appear in the model.


The methods of SDG are also the main mathematical tools used in the next paper in application to the pursuit of quantum gravity (QG). J. Król in his article \textit{Aspects of Perturbative Quantum Gravity on Synthetic Spacetimes} gives his own contribution to this huge endeavor. In the introduction, the author presents some of the results in QG obtained so far. He follows the known perturbative approach to QG (which involves an expansion of the metric tensor around the flat Minkowski spacetime and an attempt to quantize the fluctuations around it) but tries to use new methods in the context of SDG. As he notes, ``The particularly well-defined area of applicability of synthetic methods will appear to be perturbative quantum gravity'' (p.~107). Some definitions and notions are common with the previous chapter (which J. Król co-authored). The reader has to be familiar with some background knowledge and notations from the literature in order to fully follow the contents of the paper. I will not bring up the details of the article, which comprises much more considerations than are mentioned here. I only note that the author considers both the case of pure gravity (no matter fields) and the case with matter. Synthetic methods are applied in the context of the so-called BCFW procedure (from the names of the authors of \parencite{britto-2005}, also see Appendix B of the paper by Król) and lead, \textit{inter alia}, to the following results: (i) ``The pure perturbative covariant QG on synthetic spacetime is entirely finite theory [\ldots] [and] $ g_{\mu\nu} $ can now be considered as fully quantized field'' (p.~112); (ii) ``supergravity theory [for $ \mathcal{N}=8 $ and with matter fields] formulated on synthetic spacetime is again finite and $ g_{\mu\nu} $ quantized'' (p.~112). In the last section the author also reflects on the question, ``how (if at all) it can be that a perturbative QG is considered a fundamental theory'' (p.~113).

\enlargethispage{-.5\baselineskip}
The last chapter, titled \textit{Category Theory as a Foundation for the Concept Analysis of Complex Systems and Time Series}, by G.N. Nop, A.B. Romanowska and J.D.H. Smith, deals with the applications of CT to so-called concept analysis. The paper extends and generalizes the existing applications. Concept analysis studies relationships between some items and properties they possess. After the short introduction, the authors give a summary of the notions used in this context. Although they are adopted in various fields of study (and may have different names), they ``provide essentially equivalent tools for the analysis of a static system functioning at a single level'' (p.~120). To get some taste of the paper let me only mention one special notion, that of a concept of a certain context. A context is a triple made of a set of items ($ \Omega $), a set of properties ($ \Pi $), and a relation between them attributing properties to items (being a relation it is a subset of $ \Omega\times\Pi $). A concept of such a context is an ordered pair $ (A, B) $ (denoted in the text as $ (A|B) $), where $ A $ is a subset of items and $ B $ is a subset of properties, such that the set of properties common to all items in $ A $ is exactly $ B $ and at the same time the set of items attributed to all properties in $ B $ is exactly $ A $. I think one may say that in this way we get a certain characterization of some items in terms of properties (and vice versa). In general, the power sets of $ \Omega $ and $ \Pi $ (treated as poset cat\-e\-gories) are connected with certain adjoint functors known as a Galois connection. In the paper, the reader is familiarized with many other notions and some simple examples as well. Then the authors proceed to ``present the foundations for an extension of the ideas of concept analysis to changing environments, evolving complex systems, and time series analyses of successive stages of a given system'' (p.~119), which is the main goal of the paper. The extension to complex systems is done, generally speaking, in terms of certain functors from a semilattice (which serves as a kind of index of distinct levels or, if it is a chain, of time points) to appropriate cat\-e\-gories, though the procedure is more sophisticated.

In sum, the book under review is a testament to the breadth of applications of CT in physics, mathematics, and philosophy. The book is worth reading, in general, although the chapters are so diverse (in the subject matter and sometimes also in the scientific value) that I suppose the reader that is not interested in all of the interdisciplinary applications should focus on the part (s)he is interested in; e.g. chapters 7--9 may be truly engaging for readers familiar with general relativity and quantum mechanics but may be completely incomprehensible for non-physicists. The papers contained in the book are not review articles nor popular science, but are mostly conference papers, which makes the reading much more demanding (recommended for graduate-level readers and beyond) though at the same time much more fascinating.

\autorrec{Mariusz Stopa}

\end{recengenv}
