\begin{artengenv}{Colin McLarty}
	{Mathematics as a love of wisdom: Saunders Mac~Lane as philosopher}
	{Mathematics as a love of wisdom: Saunders Mac~Lane as philosopher}
	{Mathematics as a love of wisdom: Saunders Mac~Lane\\as philosopher}
	{Department of Philosophy, Case Western Reserve University}
	{This note describes Saunders Mac Lane as a philosopher, and indeed as a paragon naturalist philosopher. He approaches philosophy as a mathematician.  But, more than that, he learned philosophy from David Hilbert's lectures on it, and by discussing it with Hermann Weyl, as much as he did by studying it with the mathematically informed Göttingen Philosophy professor Moritz Geiger.}
	{naturalism, philosophy, mathematics, Aristotle.}


\vspace{3ex}
\begin{flushright}
  Go, read, and disagree for yourself.\\\parencite[p.~390]{MacLBell46}
\end{flushright}\vspace{3ex}

\lettrine[loversize=0.13,lines=2,lraise=-0.03,nindent=0em,findent=0.2pt]%
{T}{}his note describes Saunders Mac~Lane as a philosopher, and indeed as a paragon naturalist philosopher. Obviously he approaches questions in philosophy the way a mathematician would.  He is one.  But, more deeply, he learned philosophy by attending David Hilbert's public lectures on it, and by discussing it with Hermann Weyl, as much as he did by studying for a qualifying exam on it with the mathematically informed G\"ottingen Philosophy professor Moritz Geiger \parencite{McLSaundersLast,McLSaundersReck}.\footnote{Mac~Lane's close contact with Paul Bernays in G\"ottingen deserves more attention.  But my research so far has not identified strong philosophic influence from Bernays.  See Section~\ref{S:Learning}.}  Before comparing Mac~Lane to Penelope Maddy's created naturalist, the Second Philosopher, we relate him as a philosopher to Aristotle.


This is not to disagree with Skowron's view of Mac~Lane as a platonist in ontology \parencite{SkowTalk}.  This is about Mac~Lane on scientific and philosophic method.  He understood \textit{philosophy} very much as Aristotle did: Philosophy is the love of wisdom, and every science pursues wisdom and not mere facts.  I do not claim Mac~Lane got the ideas from Aristotle.  So far as I know, Mac~Lane himself had no interest in or opinion of Aristotle, though his teachers Weyl and Geiger certainly talked of Aristotle.\footnote{\textcite{BrowderMacL} give a beautiful survey of mathematics which mentions Plato and Aristotle on ontology.  However, the discussion of Plato and Aristotle summarizes the longer discussion by \textcite{BrowderRel}.}





\section{Aristotle on wisdom and science}\label{S:Aristotle}
\myquote{
  In general the sign of knowing or not knowing is the ability to teach, so we hold that art rather than experience is scientific knowledge;  some can teach, some cannot. Further, the senses are not taken to be wisdom.  They are indeed the authority for acquaintance with all individual things, but they do not tell the why of anything, for example why fire is hot, but only that it is hot.\\
  \indent It is generally assumed that what is called wisdom is concerned with the primary causes and principles, so that, as has been already stated, those who have experience are held to be wiser than those who merely have any kind of sensation, the artisan than those of experience, the craft master than the artisan.  (Aristotle, Metaphysics 981f.)\footnote{Translations of Aristotle here use ``know'' for \textit{oida}, ``scientific knowledge'' for \textit{episteme}, and ``acquaintance'' for \textit{gnosis}.  Of course ``wisdom'' is \textit{sophia}.}
}

Here Aristotle says scientific knowledge is better than acquaintance, or even experience, in two related ways:
\begin{enumerate}
  \item scientific knowledge can be taught, and
  \item scientific knowledge gives the why of things.
\end{enumerate}
He says wisdom deals with primary causes and principles.  And those are his touchstone for scientific knowledge: ``We do not know or have scientific knowledge of objects of any methodical inquiry, in a subject that has principles, causes, or elements, until we are acquainted with those and reach the simplest elements'' (Physics 184a).

Throughout his career Mac~Lane tied pedagogy to research and research to craft.  Among many examples see his early notes on presenting mathematical logic to university students \parencite{MacLSymbolic}, and his impassioned argument that theoretical education prepared men and women well to do the applied mathematics which he supervised in World War~II \parencite{MacLColumb,MacLReq}.

Mac~Lane also insists mathematical understanding includes knowing the reasons for a given theorem.  Some proofs of a theorem may reveal the reason, while other technically sufficient proofs will not reveal the reason. In his book for philosophers, Mac~Lane sketches proofs for major theorems from many subjects, like linear algebra, or complex analysis.  He often describes several alternative proofs for a single theorem and then eventually singles out one as giving the real reason.  That book is \parencite{MFF} and some examples are on pages 145, 189, 427, 455.

For me, though, the deepest connection between Aristotle's and Mac~Lane's loves of wisdom is how they say we gain this scientific knowledge.  Both believe in foundations, or ``first principles'' if you prefer, but neither believes we start with those. Aristotle's theoretical demand of philosophy was a theoretical and practical demand in mathematics for Mac~Lane:

\myquote{
The natural way of [getting scientific knowledge] is to start from the things which are more knowable and obvious to us and proceed towards those which are clearer and more knowable by nature; for the same things are not 'knowable relatively to us' and 'knowable' without qualification. So in the present inquiry we must follow this method and advance from what is more obscure by nature, but clearer to us, towards what is more clear and more knowable by nature. (Physics 184a)
}

Aristotle speaks of advancing from what is initially clear to us, towards what is more knowable by nature.  I am not sure if he believed there was a final point where the absolutely first principles and simplest elements are known so that they will never change.  Mac~Lane certainly did not believe it for mathematics.

From his early work on field theory \parencite{MacSchZero,MacSchNorm} and for the rest of his career Mac~Lane often worked to find more basic concepts in some part of mathematics. He and Eilenberg spent over a decade collaborating on ever broader uses of the concepts in their ``General theory of natural equivalences'' \parencite{GenTh}.  They meant that paper to be the only one ever needed on this technical concept for group theory and topology, but it became the founding paper of the whole field of category theory.

Only in the 1960s, after meeting graduate student Bill Lawvere, did Mac~Lane come to believe category theory could be a foundation for all mathematics.  Even then, precisely because of all the concrete mathematics that had gone into developing his ideas, Mac~Lane insisted this, and any foundation for mathematics, must be seen as ``proposals for the organization of mathematics'' \parencite[p.~406]{MFF}.  The optimal organization (i.e.~the optimal foundation) will change as mathematics develops, and will help advance those developments.  He warned that excessive faith in any ``fixed foundation would preclude the novelty which might result from the discovery of new form'' \parencite[p.~455]{MFF}.


\section{Mathematics as a love of wisdom}

Let us come to cases with one paradigmatically philosophical question, and one paradigmatically mathematical.  For Mac~Lane these questions are inseparable:%\vspace{2ex}
\begin{itemize}
         \item[Q$_1$] What are mathematical objects, and how do we come to know them? %\vspace{1ex}
         \item[Q$_2$] What are solutions to a Partial Differential Equation (PDE), and how do we come to know them?
       \end{itemize}
For Mac~Lane Q$_2$ can only be a specific case of Q$_1$.  For him, as for Aristotle, basic questions of the special sciences \textit{are} philosophy.  They cannot \textit{not} be philosophy.

Let us be clear:  A mathematician can learn a textbook answer to Q$_2$ without ever asking for a philosophy behind it.  In just the same way, a philosopher can learn the currently received answers to Q$_1$ from philosophy books, without ever asking about live mathematics.  Admittedly the math textbook answers will be more stable over time than the philosophically received answers. But that is not important.  For Mac~Lane, both of those ways of learning are failures of understanding.  They are failures of \textit{philosophy}.  For him, an answer to either one of those questions can only be valuable to the extent that you can see what it is \textit{good for}---for Mac~Lane that cannot be either a purely technical mathematical question or a purely academic philosophic one.  Think back to his work in World War II.

Mathematicians speak of solutions to PDEs in many ways:
\begin{itemize}
  \item Smooth (or, sufficiently differentiable) function solutions.
  \item Symbolic solutions.
  \item Generalized function solutions (of various kinds\dots).
  \item Numerical solutions\dots.
\end{itemize}
These different senses of solutions are sought in very different ways.  There are well understood relations between them, but the relations are not all obvious and in particular cases they may be quite difficult, and important, to find.

Mac~Lane's war work certainly involved relating different kinds of solutions to PDEs.  Even when an equation has a known exact solution by an easily specified smooth function, applying it also requires numerical solutions.  The worker has to choose which aspects are best handled in theory, so as to direct and optimize the calculations, and when best to leave theory and begin calculating.  Those choices are rarely textbook work. They are often not clear cut at all.  They require exactly what Aristotle called the wisdom of the craft master. Namely, they require grasping the \textit{why} of each kind of solution.  They require knowing not only the technical definition of each kind of solution, but \textit{what good} each one is, and especially \textit{the good} of their relations to one another.

The craft master, having wisdom, knows the whys, can teach them, and supervise work with them.

As I write this, I imagine some practice-minded philosopher challenging: ``How are philosophies like logicism, formalism, and intuitionism going to help anyone solve or apply a PDE?'' Indeed. This is why Mac~Lane so often deprecates those philosophies.  But just to give one example, Mac~Lane argued that formalism in Hilbert's hands was a step towards programmable computers. See Section~\ref{S:Learning}.  Those unquestionably help solve PDEs.

\section{The philosophy of mathematicians\\in 1930s G\"ottingen}\label{S:Learning}

\myquote{
\textit{Wir d\"urfen nicht denen glauben, die heute mit philosophischer Miene und \"uberlegenem Tone den Kulturuntergang prophezeien und sich in dem Ignorabimus gefallen. F\"ur uns gibt es kein Ignorabimus, und meiner Meinung nach auch f\"ur die Naturwissenschaft \"uberhaupt nicht. Statt des t\"orichten Ignorabimus heiße im Gegenteil unsere Losung:
Wir m\"ussen wissen -- wir werden wissen!}%\vspace{2ex}


We must not believe those, who today, with philosophical bearing and deliberative tone, prophesy the fall of culture and accept the ignorabimus. For us there is no ignorabimus, and in my opinion none whatever in natural science. In opposition to the foolish ignorabimus our slogan shall be:
We must know -- we will know!  \parencite[p.~385]{HilbNaturerkennen}\footnote{Before we try to defend or defeat Hilbert's slogan as an assertion in academic epistemology, it is in fact the most important statement in philosophy of mathematics of the past 150 years. I use the translation by \textcite[p.~1164]{ewald2005kant}.  Hilbert's address was  broadcast on the radio and a recording is available at  \url{math.sfsu.edu/smith/Documents/HilbertRadio/HilbertRadio.mp3}.}
}
Hilbert's conclusion, \textit{Wir m\"ussen wissen -- wir werden wissen!}, is engraved on his tomb in G\"ottingen.


Mac~Lane arrived in G\"ottingen just at the time Hilbert was promoting this slogan.  I do not recall Mac~Lane quoting it in lectures or conversations.  He did not have to quote Hilbert.  Everything Mac~Lane said illustrated this faith.  As you can see in \textcite{MFF,MacLAuto}, or \textcite{McLSaundersLast}, Mac~Lane followed Hilbert's mathematizing scientific optimism, rather than the specific finitist program (often called ``formalism,'' though not by Hilbert) of Hilbert's famous \textit{On the Infinite}~\parencite*{HilbUnend}.  Mac~Lane did see specific, productive mathematical value in that program though, and rejected a criticism of it by Freeman Dyson~\parencite{MacLDyson}.

Dyson supposed Hilbert seriously meant to reduce all mathematics to formal reasoning, and said ``the great mathematician David Hilbert, after thirty years of high creative achievement[\dots]\ walked into a blind alley of reductionism.''  Specifically, Dyson claimed that Hlbert ``dreamed of'' formalizing all mathematics, solving the decision problem for this formal logic, and ``thereby solving as corollaries all the famous unsolved problems of mathematics.''  Mac~Lane replied:
\myquote{I was a student of Mathematical logic in G\"ottingen in 1931-1933, just after the publication of the famous 1931 paper by G\"odel. Hence I venture to reply [\dots].
Hilbert himself called this `metamathematics.' He used this for a specific limited purpose, to show mathematics consistent. Without this reduction, no G\"odel's theorem, no definition of computability, no Turing machine, and hence no computers [\dots]. Dyson simply does not understand reductionism and the deep purposes it can serve.
}

Mac~Lane gives a concise expert review of these issues and places them in the context he knew at the time they arose.  He insists Hilbert did not tie the slogan ``we must know, we will know'' to the decision problem.  Rather, Mac~Lane says, ``[Hilbert] held that the problems of mathematics can all ultimately be solved'' without supposing metamathematics will do it. Full disclosure: I admit that after long consideration, drawing on \textcite{SiegProg,SiegBook}, I myself am unsure exactly how Hilbert and/or Bernays intended their work on the decision problem at various times. But however that may be, Mac~Lane understood Hilbert this way.  And this is Mac~Lane's own far from reductionist faith, while recognizing reductionist methods for what they actually have achieved in mathematics.

Mac~Lane learned a lot in frequent discussions with Bernays.  But for now I have to say I see no larger trace of those discussions in Mac~Lane's philosophy than is found in the letter on Dyson.  Mac~Lane's book on philosophy of mathematics is titled \textit{Mathematics: Form and Function}.  But this is clearly ``form'' as Mac~Lane learned about it from talking with Weyl and studying under Geiger~\parencite{McLSaundersLast}. It does not refer to formalism in any sense related to Bernays.  Or, at least, so it seems to me.  The reader is encouraged to go, read, and disagree if they see something else.



\section{Naturalism}\label{S:naturalism}

The decisive feature marking Penelope Maddy's Second Philosopher as a \textit{naturalist} is that:

\myquote{[She] sees fit to adjudicate the methodological questions of mathematics---what makes for a good definition, an
acceptable axiom, a dependable proof technique?---by assessing the effectiveness of the method at issue as means towards the
goals of the particular stretch of mathematics involved. \parencite[p.~359]{MadSecBook}
}
Lots of mathematicians, and essentially all leading mathematicians, do the same.\footnote{If by axioms Maddy means specifically axioms of set theory then few mathematicians ever learn those, let alone adjudicate them. Mac~Lane is famously among those few.}

The unusual thing about Mac~Lane in this regard is that he was explicitly tasked by the US government to evaluate mathematics research and teaching methods in classified reports during World War II and publicly after that.\footnote{See \textcite{MacLFederal,MacLColumb,MacLChina} and \textcite{SteingartPhD}.}  Those reports were explicitly directed to various different specific short-term and long-term goals.  All the variety he saw, and dealt with, left Mac~Lane ever more deeply impressed with the actual unity of the whole.


Precisely that background, along with his experience as Chair of the Chicago Mathematics Department, made Mac~Lane diverge from another feature of Maddy's Second Philosopher:
\myquote{
All the Second Philosopher's impulses are methodological, just the thing to generate good science [\dots].~\cite[p. 98]{MadSecond}
}
All Mac~Lane's impulses aim at producing good science and for this reason they  are not \textit{all} methodological.

Mac~Lane, like Aristotle, knows methods alone generate no science.  Besides evaluating methods of reaching goals, at least some mathematicians must evaluate goals. For Aristotle, those should be the craft masters, the wise.  In his vivid words: ``the wise should not accept orders but give them;  nor should they be persuaded, but the less wise should''~(Metaphysics 982a).  We will see, though, Mac~Lane is less focused on command than that.  He inclines more to another passage: ``those who are more accurate and more able to teach about the causes are the wiser in each branch of knowledge'' (Metaphysics 982a).


Because of his broad experience, especially evaluating both methods and goals for mathematics, Mac~Lane cannot agree that ``the goal of philosophy of mathematics is to account for mathematics as it is practiced, not to recommend reform.'' \parencite[p.~161]{MadNat} %([1997].  
Just sticking to the mathematician philosophers we have already named: Hilbert, Weyl, and Mac~Lane all knew reform is integral to mathematical practice.  You cannot separate reform from practice if you try.  And all three made explicitly philosophic arguments for their recommended reforms along with more technically mathematical ones.\footnote{Hilbert had sweeping success with his reforms.  Among many philosophic works by and on  him see~\textcite{HilbLogGrund,HilbNaturerkennen}. \textcite{WeylKontinuum} advocated what Weyl took to be Brouwer's philosophy, while \textcite{WeylDer,WeylOf} trace his eventual, regretful conclusion that in fact Hilbert was right about this and Brouwer wrong.}   This is important for philosophy of mathematics.

The paradigm case for anti-revisionism in philosophy of mathematics is Brouwer's intuitionism.  Brouwer is by far the favorite illustration of a revisionist, and is the sole example that the \textit{Stanford Encyclopedia of Philosophy} discusses under anti-revisionism in the article ``Naturalism in the Philosophy of Mathematics'' \parencite{SEPnaturalism}:
\myquote{
The mathematician-philosopher L.E.J. Brouwer developed intuitionistic mathematics, which sought to overthrow and replace standard (‘classical’) mathematics.
}

So it is important for philosophers to understand that the problem with Brouwer, according to all our exemplars Hilbert, Weyl, and Mac~Lane, is not that he had philosophical motives.  It is that he was wrong.  Actually, for Mac~Lane, Brouwer's philosophy was at best wrong.  At worst it was ``pontifical and obscure'' \parencite{MacLSymbolic}.

Immediately upon completing his doctorate in mathematics at G\"ot\-tin\-gen, Mac~Lane put a philosophy article in  \textit{The Monist}~\parencite{MacLMonist}.  Fifty years later he wrote a book describing, as he told me, what he wanted philosophers to know about math~\parencite{MFF}.  There he asks about the large array of mathematics he surveyed:
``How does it illuminate the philosophical questions as to
Mathematical truth and beauty and does it help to make judgements
about the direction of Mathematical research?''  \parencite[p.~409]{MFF}
There is a reason he puts these questions together.

He asks about mathematical truth and beauty knowing very well that few mathematicians want to pursue the question seriously, and knowing philosophers who speak of it rarely know much of the wealth.  For Mac~Lane both of those are failures of understanding and they are nothing he means to promote.\footnote{À propos, I consider Edna St.~Vincent Millay's poem ``Euclid alone has looked on Beauty bare'' incredibly true to its topic, despite that she apparently studied no mathematics beyond school textbooks based on bits of Euclid's \textit{Elements}.}  He means to promote mathematically informed philosophic pursuit of the question of mathematical truth and beauty.  And so he does of the question on the direction of research.  He seriously means to promote philosophic thought on that.  Of course he does not see philosophic thought as the sole preserve of those with philosophy degrees.  No more does he see philosophy of math as the sole preserve of those with math degrees.   Mathematics for Mac~Lane, when pursued with full awareness of its worth, is philosophy.

\end{artengenv}
