\begin{artengenv}{Zbigniew Semadeni}
	{Creating new concepts in mathematics: 
	  freedom and limitations.
	  The case of\\Category Theory}
	{Creating new concepts in mathematics: 
		  freedom and limitations\ldots}
	{Creating new concepts in mathematics: 
		  freedom and limitations.\\
		  The case of Category Theory}
	{Institute of Mathematics, 
	  University of Warsaw}
	{The purpose of the paper is to discuss the problem of possible limitations 
	of freedom in mathematics and to look for criteria which would help us 
	to distinguish---in the historical development of mathematics---new concepts 
	which were natural follow-ups of the previous ones from new concepts which opened 
	unexpected ways of thought and reasoning. \par
	The rise of category theory (CT) is analysed, in particular, earlier ideas (which 
	were precursors of the theory) and its initial development. The question of the 
	the origin of the term \textit{functor} is discussed; the presented evidence strongly 
	suggests that Eilenberg and Carnap could have learned the term from Kotarbi{\'n}ski 
	and Tarski.}
	{categories, functors, Eilenberg-Mac Lane Program, mathematical 
	cognitive transgressions, phylogeny, platonism.}








\section{Introduction}
\lettrine[loversize=0.13,lines=2,lraise=-0.03,nindent=0em,findent=0.2pt]%
{T}{}he celebrated dictum of Georg Cantor that ``The very essence of mathematics lies 
precisely in its freedom'' expressed the idea that in mathematics one can freely 
introduce new notions (which may, however, be abandoned if found unfruitful or 
inconvenient).\footnote{The italics in the original sentence ``Das \textit{Wesen 
der Mathematik} liegt gerade in ihrer \textit{Freiheit}'' are Cantor's. 
The first version was published in 1879, reprinted in \parencite[p.34]{Cantor}, discussed 
by \citeauthor{Ferreiros} \parencite*[p.257]{Ferreiros}. } %%% koniec footnote
This way Cantor declared his opposition to claims of Leopold Kronecker who objected 
to the free introduction of new notions (particularly those related to the infinite). 

Some years earlier Richard Dedekind stated that---by forming, in his theory, a cut 
for an irrational number---we \textit{create} a new number. For him this was an example 
of a constructed notion which was a free creation of the human mind \parencite[\S~4]{Stetigkeit}. 

In 1910 Jan Łukasiewicz distinguished \textit{constructive notions} from empirical 
\textit{reconstructive} ones. He referred (with reservation) to Dedekind’s statement 
and pointed out that a consequence of our ``creation'' of those notions is the 
 spontaneous emergence of countless relations which no more depend on our will. 

Until the discovery of non-Euclidean geometries, geometry was regarded as an 
abstraction of the spatial reality. The freedom of creation in geometry was 
limited by this reality. Hilbert advocated a formal point of view, broadening the 
freedom of choosing the axioms, while Poincar\'e maintained that the axioms of 
geometry are conventions.
 
Clearly, the freedom of mathematics is limited by logical constraints. At the same 
time, logical inference yields deep meaning to mathematics. As Michał Heller put it, 
``If I accept one sentence, I must also accept another sentence. Why must I? 
Who forces me? 
Nobody. Yet, I must. Generally, we bear badly any restrictions of our liberty, but 
in the case of mathematical deduction inevitability of the conclusion gives us the 
feeling of safety (I have not deviated from the way) and of the accompanying 
intellectual comfort, sometimes even great joy'' \parencite[p.21]{Heller}. 

A mathematician trying to prove a theorem knows the feeling of an invisible wall 
which blocks some intended arguments. Also new concepts must be consistent with 
earlier ones and must not lead to contradiction or ambiguity. Moreover, in 
mathematical practice, only intersubjective mental constructions are accepted.  

The purpose of this paper\footnote{The present paper is based on a talk delivered 
at XXIII Kraków Methodological Conference 2019: {\it Is Logic a Physical Variable?}, 
7-9 November 2019.} % koniec footnote
is to look for restraints and patterns in the historical development of 
mathematics.\footnote{ The significant question of degree of freedom in mathematical 
conceptualization of physical reality is not considered here.}  % koniec footnote
Some types of paths will be distinguished, first generally, in the context of 
the historical development of mathematics, and then they will be used to highlight 
some features of the rise of category theory (CT). 

Michael Atiyah, in his \textit{Fields Lecture} at the World Mathematical Year 2000 
in Toronto, expressed his view that ``it is very hard to put oneself back in 
the position of what it was like in 1900 to be a mathematician, because so much 
of the mathematics of the last century has been absorbed by our culture, by us. 
It is very hard to imagine a time when people did not think in those terms. In fact, 
if you make a really important discovery in mathematics, you will then get omitted 
altogether! You simply get absorbed into the background'' \parencite[p.1]{Atiyah}. 

This statement may be appear startling, as the mathematical meaning of a text from 
around 1900 is generally believed to be time-proof. Yet what Atiyah had in mind was 
not the meaning of published texts---definitions, theorems and proofs---
but the way mathematicians thought at that time. 

It was 75 years ago when the celebrated paper by 
%Samuel Eilenberg and Saunders Mac Lane 
\citeauthor{E-ML} was published.\footnote{Mac Lane earlier in his life (in particular 
in \parencite{E-ML} and \parencite{Duality} wrote his name as MacLane. Later he 
began inserting a space into his surname, in particular in \parencite{Working}. } 
This event marked the rise of category theory (CT). As the 
development of mathematics accelerated in the second half of the 20\textsuperscript{th} century, 
it may be hard to fully imagine how mathematicians thought in 1945. Their 
definitions, theorems and comments are clear, but one should be aware that a 
reconstruction of their ideas, their thinking may be specifically biased by our 
present understanding of the mathematical concepts involved. 

\section{Background conceptions}
The main ideas of this paper---which includes a very wide spectrum 
of examples, from ancient Greek mathematics to modern, from children’s counting 
to CT---are the following: 
\begin{itemize} 
\item A mathematical concept, no matter how novel, is never independent of the 
previous knowledge; it is based on a reorganization of existing ideas. 
\item A radically new mathematical idea never germinate in somebody's mind 
without a period of incubation, usually a lengthy one. 
\item There is a long distance to cover between: \par 
(1) a spontaneous, unconscious use of a mathematical idea or structure in a concrete 
setting, \par 
(2) a conscious, systematic use of it. 
\item A person who has achieved a higher level of mathematical thinking is often 
unable to imagine thinking of a person from another epoch or of a present learner 
and, consequently, may unconsciously attribute to him/her an inappropriate (to high 
or too low) level of thinking.  
\end{itemize}

\subsection{Transgressions}
A \textit{mathematical cognitive transgression} (or briefly: a \textit{transgression}) 
is defined as crossing---by an individual or by a scientific community---of 
a previously non-traversable limit of own mathematical knowledge or 
of a previous barrier of deep-rooted convictions. Moreover, it is assumed that: \par
\begin{enumerate} 
\item the crossing concerns a (broadly understood) mathematical idea and the difficulty 
is inherent in the idea, 
\item the crossing is critical to the development of the idea and related 
concepts, 
\item it is a passage from a specified lower level to a new specified upper level, 
\item the crossing is a result of conscious activity (the activity need not 
be intentional and purposely orientated towards such crossing; generally such 
effect is not anticipated in advance and may even be a surprise). 
\end{enumerate} 

In the history of mathematics there were numerous transgressions, of different 
importance, some great ones and many ``mini-transgressions''. Usually they were 
not single acts---they involved a global change of thinking which matured for 
years or even generations and they based on the work of many people. In ancient 
times two transgressions were the most significant: 
\begin{itemize}
\item The transition from practical dealing with specific geometric shapes to deductive 
geometry.  
\item The celebrated discovery (in the 5\textsuperscript{th} century BC) that the diagonal of a 
square is incommensurable with its side, that they are 
\textgreek{'alogos} 
(\textit{a-logos}), without a ratio, irrational (in modern setting, foreign to Greek 
thinking, it was the irrationality of $\sqrt{2}$) \cite[pp.80-81]{Baszmakowa}. 
The Pythagorean paradigm was undermined by this \textgreek{apor'ia}.
Their understanding of mathematics was eroded. However, nothing certain is 
known about this discovery. Stories presented in popular books are based on 
doubtful legends from sources written seven or eight centuries later 
\parencite[p.21, 51]{Knorr}. This incommensurability could not be a single 
discovery by an individual. It must have been a lengthy process. Never in the 
historical development of mathematics such a major change occurred in short time. 
\end{itemize}

\noindent In modern history there were many transgressions. Let us list some of 
the best known: 

\begin{itemize}
\item Acceptance of negative numbers.
\item The transition from potential infinity (infinity at a \textit{process} 
level) to the \textit{actual} infinity (infinity as an object). 
\item The emergence of projective geometry.  
\item The discovery of non-Euclidean geometries. 
\end{itemize}

\noindent We will discuss the case of CT, arguing in particular that creating the 
theory of \textit{elementary topoi} in CT should be regarded as a major transgression. 

\subsection{Phylogeny and ontogeny}
The term \textit{phylogeny} refers here to the evolutionary history of mathematics 
(or rather to its modern reconstructions), from ancient times on. 
\textit{Ontogeny} means the development of basic mathematical concepts and 
structures in the mind of an individual person, from early childhood. Phylogeny and 
ontogeny are in some sense complementary descriptions \parencites[][]{HF}[][pp.4-29]{P-G}.
%% Freudenthal, Piaget-Garcia 

In case of mathematics, the oft-quoted phrase: \textit{ontogeny recapitulates 
phylogeny} implies that one can learn from the history of old mathematics for the 
sake of present teaching. This sometimes gives useful hints, e.g. one may argue that 
since the historical process of forming the general concept of a function took 
centuries (from Descartes, if not much earlier, to Peano and Hausdorff), we should 
not expect that a secondary school student can grasp it---learning Dirichlet’s 
description---after a few lessons. The general concept of the function was not 
yet quite clear to mathematicians of the first half of the 19\textsuperscript{th} century 
\parencites[][Appendix 2]{Lakatos}{Youschkevitsch}[][pp.27--30]{Ferreiros}. 

On the other hand, the idea that ontogeny recapitulates phylogeny may be misleading. 
Piaget always stressed that arithmetic cognition results from logico-mathematical 
experience with concrete objects, pebbles say, and is educed from \textit{the 
child’s actions} rather than from heard words. It is abstracted from a coordination of 
intentional motions and accompanying thoughts. Nevertheless, Piaget was in favour 
of the phylogeny-ontogeny parallelism and reasoned roughly as follows. Since 
one-to-one correspondence preceded numerical verbal counting in the very early 
periods of human civilization (evidenced on artefacts such as notched bones and 
also found in rude unlettered tribes), the same should apply to children. Cantor’s 
theory of cardinal numbers confirmed this thinking \parencite[pp.259--260]{B-P}.
Consequently, Piaget and many educators insisted on one-to-one correspondence 
as a foundation of early school arithmetic, neglecting the fact that nowadays 
children learn number names early, often together with learning to speak, and 
moreover counting is now deeply rooted culturally. Research evidence shows that 
counting, rather than one-to-one correspondence, is a basis of the child’s 
concept of number \parencite[pp.77--82]{G-G}.
In this way the phylogeny–ontogeny parallelism adversely affected early mathematics 
education in the time of the `New Math' movement. 

Hans Freudenthal, in the context of mathematics, suggested the converse idea: 
\textit{What can we learn from educating the youth for understanding the past of 
mankind?} \,This reverses the traditional direction of inference in the 
phylogeny–ontogeny parallelism. In particular, one may ask whether contemporary 
knowledge of the difficulties in the transition from the concrete to more abstract 
mathematical reasoning of children may be helpful in better understanding of 
limitations of our reconstructions of the development of the early Greek mathematics.  

In the sequel, certain aspects of the development will be traced both in phylogeny 
and in ontogeny, inextricably % [nierozdzielnie] 
intertwined with the the mathematical questions themselves. 

\subsection{Platonizing constructivism in mathematics}
The theoretical framework of the paper is platonizing constructivism in mathematics. 
It is assumed that: 
\begin{itemize} 
\item each of the three major positions in the philosophy of mathematics from the beginning of 
the 20th century: \textit{platonism}, \textit{constructivism}, \textit{formalism} 
describes some inherent, complementary features of mathematics; 
\item they can be reconciled provided that they are regarded as \textit{descriptive}, 
as an account of some inherent features of mathematics, and \textit{not normative}, 
i.e., when one skips the eliminating words (as `only', `oppose') which explicitly deny 
other standpoints. Moreover, various versions of the three positions often overlap. 
\end{itemize}

\noindent In the sequel, the term \textit{constructivism} will not be understood as in papers on 
foundations of mathematics, but rather in a way akin to its meaning in research on mathematics 
education, related to post-Piagetian psychological versions of constructivism. 
Briefly, one assumes here that humans construct mathematical concepts in their minds  
and discover their properties. A~concept develops its necessary structure 
as a consequence of its context and---in the long term---becomes 
\textit{cristalline} in the sense of David \citeauthor{Tall} \parencite*[p.27]{Tall}; then its 
properties appear independent of our will. 
This phenomenon may be traced both in phylogeny and in ontogeny. 
Moreover, in each essential progression, new mental structures are build on the 
preceding ones and are always integrated with previous ones \parencite[pp.22--29]{P-G}.  

By platonizing constructivism we mean an analysis---in constructivistic terms 
--- of sources and consequences of the platonistic attitude of a majority of 
mathematicians and contrasting them with the well-known difficulties of consistent 
platonism in the philosophy of mathematics. 

\section{Developmental successors}
After the introductory examples we now look for ways to distinguish between: 
\begin{itemize}
\item mathematical concepts which---historically---were natural successors to 
previous ones,  
\item concepts which could be conceived and defined only 
after opening new paths of thought and reasoning. 
\end{itemize}

\noindent The following metaphorical labels will be used: \textit{onward development}, 
\textit{branching-off},    %%% \textit{merging}, 
\textit{upward development}, \textit{downward development}, interpreted with 
examples. We do not expect to find clear criteria, but the ensuing discussion 
may be illuminating. 

The \textit{emergence of numerals} in the Late Stone Age is evidenced by tally 
marks (in the form of notched bones). Ethnologists have found that early tribes 
had only two counting words: \textit{one} and \textit{two}, followed by \textit{many}. 
Also in present Indo-European languages these two numbers and their ordinal 
counterparts are linguistically different from the following numbers. 
It has been suggested that the proto-Indo-European number *\textit{trei} (three) was 
derived from the verb *\textit{terh} (meaning: \textit{pass}); thus, the word 
\textit{three} is related to \textit{trans}. This may be a hint of a very ancient 
mental obstacle between numbers \textit{two} and \textit{three}. 

One may conjecture that \textit{after the passage from 2 to 3 there was no notable 
obstacle to gradual development of unlimited counting}. Of course, the actual 
development took centuries, if not millennia. Anyway, for present children 
there is no hurdle between 2 and 3, as they are taught counting very early. 
Moreover, counting starts to make sense with three items. 

\subsection{Onward development}
Onward development of indefinite counting includes its \textit{developmental 
successors}: simple addition of natural numbers (which develops through a stage 
called \textit{count all} and then a more advanced stage \textit{count on}), 
subtraction (as taking away), multiplication, division (originally there are two 
kinds of it: \textit{equal sharing} and \textit{equal grouping}), and even simple 
powers, all within some range of natural numbers.  

These concepts are included in the onward development of counting, by virtue of 
the following features: 

\begin{itemize}
\item \textit{no branching}: each new concept naturally comes after the previous ones; 
\item \textit{ontological stability}: each concept (e.g., number 17, product 
\mbox{$3\times 6$}), remains essentially the same object, although the related 
ideas are enriched after each extension of the scope of arithmetic and---in the 
historical development---are subject to evolutionary changes.
\end{itemize}

The conception of developmental successors, outlined here, does not take into account 
a relative difficulty of concepts; what is crucial is whether they follow the previous 
lines of thinking. 

\subsection{Branching-off}
This conception arises from a negation of the first requirement in the description 
of an onward development. An example of it are fractions, which \textit{branch off} 
from natural numbers; it is not onward development, although there are many ties 
between natural numbers and fractions. 

There are two ways of introducing fractions to children. In the first, some 
idealized whole is divided into $m$ equal parts and then $n$ of them are taken. 
In the second, $n$ whole things are equally divided into $m$ parts. 
They are two main aspects of the concept of a fraction. For instance, $\frac{3}{4}$ 
of pizza may be obtained by cutting it into 4 parts and taking 3 such parts 
(thus $\frac{3}{4} = 3 \times \frac{1}{4}$). A more advanced way of thinking of 
$\frac{3}{4}$ is 3 divided by 4; the latter may be explained with the example 
of 3 pizzas to be divided among 4 persons. 

The distinction looks quite elementary. Yet, it was significant in the phylogeny 
of fractions. First procedure is akin to that of ancient Egyptians, the 
second -- to Greek ratios; both were inherited by Arabic mathematicians. In the 
ontogeny the two ways are always present, but not necessarily noticed. The 
following reminiscence by William Thurston (1946-2012), written 8 years after 
he had received Fields Medal, describes his discovery of the identification of 
previously different objects. 

\myquote{
I remember as a child, in fifth grade, coming to the amazing (to me) realization 
that the answer to 134 divided by 29 is $134/29$ (and so forth). What a tremendous 
labor-saving device! To me, `134 divided by 29' meant a certain tedious chore, 
while $134/29$ was an object with no implicit work. I went excitedly to my father 
to explain my major discovery. He told me that of course this is so, $a/b$ and 
$a$ divided by $b$ are just synonyms. To him it was just a small variation in 
notation \parencite[p.848]{Thurston}.
}

\noindent The fraction $\frac{a}{b}$ becomes identified with the result of division 
$a\div b$ and---in this synthesis---they both form a single mathematical object. 
Philosophically, however, it is an ontological change: two different beings, results 
of two different mental constructions, become regarded as a single one. 

Onward branching-off can be traced in many parts of mathematics.  
Calculus branches off from a theory of the field of real numbers (axiomatic or 
based on a construction). 
Infinite sequences of real numbers branch off from elementary algebra of real numbers. 
Limits of sequences branch off from general theory of sequences. 
These examples vividly show that the question of distinguishing branching-off from 
onward development is delicate, as the criteria are far from being precise, but it 
may contribute to better understanding the historical development of mathematics. 

\subsection{Upward development and downward development}
By \textit{upward development} of a piece of mathematics we mean passing from some 
concepts and relations between them to a more abstract version of them. Examples: 
\begin{itemize}
\item Transition from practical addition (verbalized as, e.g., \textit{two and 
three make five}) to symbolic version (e.g., 2+3=5) took centuries (the sign $+$ 
appeared in some 15\textsuperscript{th} century records; the first occurrence of the equality 
sign $=$ was found in a text by a Welsh mathematician Robert Recorde from 1557). 
\item Transition from $\mathbb{R}^n$ 
to an axiomatically given vector space over $\mathbb{R}$.  
\item Transition from a vector space over $\mathbb{R}$ 
to a vector space over a field. 
\item Transition from group theory to the category \textbf{Grp}. 
\item Transition from a category to a metacategory (in the sense 
of \parencite[pp.7--11]{Working}). 
\end{itemize}

\textit{Downward development} is---in some sense---an inverse process, 
to more concrete questions or to a lower level of abstraction, so the above examples 
may be used the other way round. Also some typical applications of mathematics 
may be included here, e.g., the passage from abstract Boolean algebras to a 
description of certain types of electrical circuits (as conceived by Claude \citeauthor{Shannon_symbolic_1936}~\parencite*{Shannon_symbolic_1936}).

In the 20\textsuperscript{th} century the mathematics grew rapidly and the upward development became 
much easier mentally as a result of both: a general change of the attitude of 
mathematicians toward abstraction and the routine of expressing all concepts in 
the language of set theory. 
Branching-off were so frequent that the above metaphors are of little use. 
There is, however, a notable exception: a new theory which opened a new direction 
of thinking, so its beginnings may be discussed in a way akin to that used with 
respect to distant past.   

\section{The rise of category theory (CT)}
A very special feature of CT is that it has a pretty precise date of its official 
birth: the publication of the paper by Samuel Eilenberg and Saunders Mac Lane 
\parencite*{E-ML}. It was presented at a meeting of the American Mathematical Society in 
1942 and published in 1945.

According to the Stanford Encyclopedia of Philosophy, \textit{CT ``appeared almost out 
of nowhere''}. Not quite so. As in any mathematical theory, some CT ideas had been 
conceived much earlier, particularly in algebraic topology, and some of them can 
be traced to the 19\textsuperscript{th} century. 

Many conceptual transformations---either explicit and well recognized or 
used implicitly, without awareness---contributed to the rise of CT. Going back,  
a significant factor was the historic development of the mathematical concept of 
a function. 

Until the beginning of the 19\textsuperscript{th} century, a general symbol for a function ($f$ or 
$\varphi$) was almost non-existent \parencite{Youschkevitsch}. 
A significant step toward CT was the general notion of a mapping introduced by 
Dedekind in 1888. In his \textit{Enkl{\"a}rung} (\textit{explanation}) 
he did not define the concept of \textit{Abbildung} $\varphi$ (literally: 
\textit{image} or \textit{representation}) from a set (\textit{System}) $S$ into a set 
$S^\prime$, but interpreted it generally as an arbitrary law (\textit{Gesetz}) according 
to which to each element $s$ there corresponds (\textit{geh{\"o}rt}) a certain 
thing (\textit{Ding}) $S^\prime = \varphi(s)$, called the image (\textit{Bild}) 
of~$s$. He also defined a composed mapping (\textit{zusammengesetzte Abbildung}) 
$\vartheta(s) = \psi(s^\prime) = \psi(\varphi(s))$ of two given ones, denoted 
as $\varphi.\psi$ or $\varphi\psi$, defined injective mappings (\textit{{\"a}hnlich} 
or \textit{deutlich}), the inverse mapping and proved their main properties 
\parencites[][\S2--4]{Was_sind}[][p.88--90, 228--229]{Ferreiros}.

Emmy Noether in her lectures in the 1920s emphasised the role of homomorphisms 
in group theory. Before her, groups were understood as generators and relations 
(in modern terms, as quotients of free groups). She also argued that the homology 
of a space is a group, is an algebraic system rather than a set of numbers assigned 
to the space. Her lectures and the lectures of Emil Artin formed a basis for the 
celebrated book by van der Waerden \parencite*{Waerden} on modern algebra. This current 
of thought led to CT.

Generally, in the symbol of the type $f(x)$, the part $f$ was always understood 
as fixed and $x$ was a variable. At the end of the 1920s, however, in functional 
analysis and related fields, a new way of thinking emerged. In certain situations 
the roles of symbols in $f(x)$ reversed: the point $x$ was regarded as fixed while 
the function $f$ became a variable (as an element of a function space), e.g., 
$x\in [0,1]$ was fixed and $f$ was a variable in the space $C([0,1])$ of continuous 
functions on the interval $[0,1]$. In this new role, the point $x$ became a functional 
$\delta_x$ on $C([0,1])$. Such a change of the roles function-element became crucial, 
e.g., in the Potryagin duality of locally compact abelian groups \parencite{Hewitt} 
and in Gelfand’s theory of commutative Banach algebras. It was also used by \citeauthor{E-ML} 
in their first example (finite dimensional vector spaces 
and their dual spaces) motivating the concept of a natural equivalence.  

A crucial example of a contravariant functor was the adjoint $T^\ast$ of a linear 
operator $T$ on a Hilbert space, introduced in 1932 by John von Neumann.\footnote{\citeauthor[p.330]{Century} tells a story how Marshall Stone advised von Neumann to introduce the symbol $T^\ast$ and how it changed the publication. He also 
mentions a fact which may interests philosophers: in 1929 von Neumann lectured in 
G{\"o}ttingen and presented his axiomatic definition of a Hilbert space, while David 
Hilbert---listening to it---evidently thought of it as of the concrete 
space $\ell^2$, not in the axiomatic setting.} % koniec footnote

According to Mac Lane, abstract algebra, lattice theory and universal algebra were 
necessary precursors for CT. However, he also suggested that certain notational 
devices preceded the definition of a category. One of them was the fundamental idea 
of representing a function by an arrow $f\colon X\to Y$, which first appeared in 
algebraic topology about 1940, probably introduced by Polish-born topologist Witold 
Hurewicz \parencites[][p.29]{Working}[][p.333]{Century}. Originally, it looked as just 
another symbol, but from a later perspective the use of such symbol was one of the 
key changes. Thus, a notation (the arrow) led to a concept (category). Such new symbols 
later got absorbed into the background of mathematical thinking, used as something 
obvious. Together with commutative diagrams, which were probably also first used by 
Hurewicz, they paved the way to CT. 

Mac Lane often accented two features of mathematics: computational and conceptual. 
He noted that the initial discovery of CT came directly from a problem of calculation 
in algebraic topology \parencite[p.333]{Century}.

Eilenberg and Mac Lane were aware that they introduced very abstract mathematical 
tools, which did not fit any algebraic system in the Garrett Birkhoff's universal 
algebra. It might seem too abstract and was certainly off beat and a ``far out'' 
endeavour. Although it was carefully prepared, it might not have 
seen the light of day \parencite[p.130]{MLonE}.

\subsection{The origin of the term \textit{functor} }
Mac Lane has written ``\textit{Categories, functors, and natural transformations 
were discovered by Eilenberg–Mac Lane in 1942}'' \parencite[p.29]{Working}. The word 
``discovered'' may be regarded as an indication of a hidden Platonistic attitude 
of Mac Lane, in spite of his verbal declarations against
Platonism \parencites[][pp.447--449]{Form}[][]{Krol}[][]{Skowron}. He also wrote: 

\myquote{
  Now the discovery of ideas as general as these is chiefly the willingness to 
  make a brash or speculative abstraction, in this case supported by the pleasure 
  of purloining words from the philosophers: ``Category'' from Aristotle and Kant, 
  ``Functor'' from Carnap (\textit{Logische Syntax der Sprache}) 
  \parencite[pp.29--30]{Working}.
}

\noindent This sentence has been taken very seriously by several authors. However, 
the way it was phrased suggests that it was rather intended to be a delicate 
joke.\footnote{ Let us note that the noun \textit{brash} means a mass of fragments; 
according to \textit{Cambridge International Dictionary of English} (1995), the 
adjective \textit{brash} is disapproving, referring to people who show too much 
confidence and too little respect, while \textit{Webster's New World Dictionary} 
(1984) lists---as meanings of \textit{brash}---also \textit{hasty and reckless}, 
\textit{offensively bold}. On the other hand, the word \textit{purloining} means 
\textit{stealing} or \textit{borrowing without permission}. Such a comment (with 
the word \textit{pleasure}) by Mac Lane concerning CT could not be serious. On the 
other hand, in 2002 Mac Lane came back to Carnap, adding: 
``Also the terminology was largely purloined: “category” from Kant, “natural” 
from vector spaces and “functor” from Carnap. (It was used in a different sense in 
Carnap’s influential book \textit{Logical Syntax of Language}; I had reviewed the English 
translation of the book (in the Bulletin AMS 1938) and had spotted some errors; 
since Carnap never acknowledged my finding, I did not mind using his 
terminology)'' % end quote ital.
\parencite[pp.130--131]{MLonE}.} % end of footnote
Attributing the origin of the term \textit{category} to Aristotle and Kant is clear, 
although in 1899 Ren\'e-Louis Baire (in his \textit{Th\`ese}) introduced---in 
another context---the word \textit{category} to 
mathematics.\footnote{ A subset $A$ of a topological space $X$ is called \textit{a 
set of first category} (\textit{un ensemble de pre\-mi\`ere cat\'e\-gorie}) in $X$ 
iff $A$ is the union of a countable family of nowhere dense sets; otherwise it is 
\textit{a set of the second category} (\textit{Menge erster und zweiter Kategorie}). 
The celebrated \textit{Baire category theorem} states, in a generalized form, 
that a complete metric space is not a set 
of the first category \parencites[][p.328]{Hausdorff}[][\S~10]{Kuratowski}. The 
clumsy term \textit{set of the first category} was later replaced by the term 
a~\textit{meager set} \parencite[p.201]{Kelley}. In the 1930s Baire category theorem 
was a very popular tool in the Warsaw school of topology, so Eilenberg must have 
known it.} % end of footnote
Concerning the origin of the term \textit{functor} in CT, one may recall the 
following facts:\footnote{ The author is indebted to Professor Jan Wole\'nski 
for the relevant information.} 

\begin{itemize}
\item In Rudolf Carnap's book \textit{Abriss der Logistik} \parencite{Abriss} the term ``Funktor'' does not appear. 
\item In 1929 the Polish term ``funktor'' was used in propositional calculus by 
Tadeusz Kotarbi\'nski in his book \parencite{Kotarbinski}.
\item Alfred Tarski often emphasised that Kotarbi\'nski had been his teacher 
\parencite[part~2]{Feferman}. 
\item Carnap met Tarski in Vienna in February 1930 and visited Warsaw in November 
1930; he learned much from Tarski. 
\item In 1933 Tarski, in the Polish version of his famous paper \textit{On the 
concept of truth in formal languages} \citeauthor{Tarski_sem} \parencite*{Tarski_sem} used the term ``funktor'' 
and mentioned that he owed the term to Kotarbiński.  
\item Carnap used the term ``Funktor''  in his book \parencite*{Logische} (quoted 
by Mac Lane) in a sense more general than that of Kotarbi\'nski and Tarski. 
\item Eilenberg studied mathematics in Warsaw from 1930. In 1931 he attended 
Tarski's lectures on logic \parencite[part~3 and~12]{Feferman}. He left Warsaw in 1939. 
\end{itemize}

This evidence strongly suggests that both Carnap and Eilenberg could have learned, 
independently, the term from Kotarbi{\'n}ski and Tarski.

\subsection{Was the original CT an onward development of previous mathematical theories?}
Using a metaphor explained above, one may argue that the definition of a 
category and of a functor were within a major onward development of part of 
mathematics of the first half of the 20th century, that is set theory, 
algebra, topology etc. Indeed, for a person working in group theory, say, 
a natural continuation should be to think of all groups, their homomorphisms, 
isomorphisms, and the composites as of a single whole: \textit{group theory}. 
Similarly one could think of vector spaces with linear maps as of another whole. 
Some analogies between theories were obvious. Moreover, axioms of CT are 
reminiscent of those of semigroup theory. The concept of a covariant functor 
was a natural analogue of homomorphisms of algebras. Contravariant functors 
had been present in various duality theories (e.g., in Pontryagin’s duality 
mentioned above). CT provided general concepts applicable to all branches of abstract 
mathematics, contributed to the trend towards uniform treatment of different 
mathematical disciplines, provided opportunities for the comparison of 
constructions and of isomorphisms occurring in different branches of mathematics, 
and may occasionally  suggest new results by analogy \parencite[p.236]{E-ML}. 

The great achievement of Eilenberg and Mac Lane was the idea that a formalization of 
various evident analogies was worth systematizing and publishing. The initial neglect 
of \parencite{E-ML} by mathematicians was very likely a result of the fact that it was 
regarded as a long paper within onward development of known part of mathematics, with 
many rather simple definitions and examples, tedious verification of easy facts, 
and no theorem with an involved proof. Ralf Kr{\"o}mer, in his book on the history 
and philosophy of CT, has outright stated that Eilenberg and Mac Lane needed to 
have remarkable courage to write and submit for publication the paper almost 
completely concerned with conceptual clarification \parencite[p.65]{Kromer}. 

A novelty of \parencite[p.272]{E-ML}, which at first appeared insignificant, was regarding 
elements $p_1, p_2$ of a single quasi-ordered set $P$ as objects of a category, 
with a unique morphism $p_1 \to p_2$ iff $p_1 \le p_2$ and no morphisms otherwise. 
This opened a way to a series of generalizations, in particular regarding certain 
commutative diagrams as functors on small categories. 

One may argue that this achievement was still within onward development of CT 
as it was within the scope of previous knowledge. Let us recall that the 
difficulty and the originality of a theorem are not taken into account; what is 
crucial is whether the concepts involved are natural extension of the previous 
knowledge and thinking. 

The category axioms represent a very weak abstraction \parencite[p.25]{Goldblatt}. 
In spite of this fact, a few years later the conceptual clarification turned out 
highly effective in the book \textit{Foundations of Algebraic Topology} written 
by Eilenberg together with Norman Steenrod \parencite*{Steenrod}. The latter admitted in a 
conversation that the 1945 paper on categories had a more significant impact on 
him than any other research paper, it changed his way of thinking. 

\subsection{The Eilenberg-Mac Lane Program}
This program has been formulated as follows: 
\myquote{
The theory also emphasizes that, whenever new abstract objects are 
constructed in a specified way out of given ones, it is advisable to regard 
the construction of the corresponding induced mappings on these new objects 
as an integral part of their definition. The pursuit of this program entails 
a simultaneous consideration of objects and their mappings (in our terminology, 
this means the consideration not of individual objects but of categories). [...] 
\par The invariant character of a mathematical discipline can be formulated in 
these terms. Thus, in group theory all the basic constructions can be regarded 
as the definitions of co- or contravariant functors, so we may formulate the 
dictum: The subject of group theory is essentially the study of those constructions 
of groups which behave in a covariant or contravariant manner under induced 
homomorphisms. More precisely, group theory studies functors defined on well 
specified categories of groups, with values in another such category. This may be 
regarded s a continuation of the Klein Erlanger Programm, in the sense that a 
geometrical space with its group of transformations is generalized to a category 
with its algebra of mappings. \par 
[...] such examples as the ``category of \textit{all} sets'', the ``category of 
\textit{all} groups are illegitimate. The difficulties and antinomies are exactly 
those of ordinary intuitive \textit{Mengenlehre}; no essentially new paradoxes are 
involved. [...] we have chosen to adopt the intuitive standpoint, leaving the reader 
free to insert whatever type of logical foundation (or absence thereof) he may prefer. 
[...] \par It should be observed first that the whole concept of a category is 
essentially an auxiliary one; our basic concepts are essentially those of a functor 
and of a natural transformation. [...] The idea of a category is required only by the 
precept that every function should have a definite class as domain and a definite 
class as range, for the categories are provided as the domains and ranges of functors.
\parencite[p.236--237, 246--247]{E-ML}
}

The quoted comparison of CT to the celebrated program of Felix Klein shows vividly 
that the authors regarded their work as significant. 
A important novelty of \parencite{E-ML} was to use the same letter to denote both: 
the object component of a functor and its morphism component. This had not been 
a common practice, even when both correspondences were dealt with in a single 
paper. This novelty and the whole program fit well Atiyah's conception (quoted 
above) of ideas \textit{absorbed by mathematicians' culture}. 

CT is both a specific domain of mathematics and at the same time a conceptual framework 
for a major part of modern theories. 

\section{The amazing phenomenon of unexpected branchings-off in CT}
Up to this point CT might be regarded are being within onward development of 
earlier theories: algebra, topology, functional analysis. However, a~branching-off 
in \parencite{E-ML} is the concept of a \textit{natural equivalence} (central in the title 
of the paper), with an essential use of commutative diagrams.\footnote{ It is not 
clear why Eilenberg and Mac Lane refrained from setting the concept in the general 
form of a \textit{natural transformation} (examples abounded). Perhaps they felt 
they should not pursue a still more general setting without accompanying 
results.} % end of footnote
It was a completely new idea, but its initial impact was limited. 

After 1945 CT---as a general theory---lay dormant till the emergence of 
significant new concepts and a breaking series of major branchings-off in the 
second half of the 1950s. One of their outstanding features was a new type of a 
definition, formulated in the form of a \textit{unique factorization problem}. 
First such explicit definition \parencite*{Samuel} appeared in the paper by Pierre Samuel, 
a member of the Bourbaki group, on free topological groups, albeit it was still 
in the language of set theory, without arrows. 
In \cite*{Duality}, commutative diagrams were demonstrated as a convenient 
tool in such problems by  \citeauthor{Duality}. 
The concept of a \textit{dual category}, formulated in \parencite[p.259]{E-ML} and 
further developed by \citeauthor{Duality} \parencite*{Duality}, had its conceptual roots in various 
duality theories, particularly in that of projective geometry. The \textit{product} 
of two categories was an analogue of that for groups and various algebras. 
Mac Lane analysed the concept of duality, 
stressed diagrammatic dualities of various pairs of concepts and presented the 
definitions of \textit{direct} and \textit{free products} in group theory (later 
generalized to the concepts of categorial \textit{products} and \textit{coproducts}, 
respectively). 

\subsection{A metamorphosis from Eilenberg--Mac Lane Program to mature CT}
A turning point in the development of CT was the seminal paper by Daniel \citeauthor{Kan} \parencite*{Kan}
on \textit{adjoint functors}. In much the same time period, independently, 
several closely related concepts and results were worked out: \textit{representable 
functors}, \textit{universal morphisms}, \textit{Yoneda lemma}, various types of 
\textit{limits} and \textit{colimits} \parencite[pp.345--352]{Century}. Special cases of 
them had a long earlier history in specific situations in algebra and topology 
(e.g., Freudenthal's theorems on \textit{loops} and \textit{suspensions} in homotopy 
theory proved in 1937). This confirms a known phenomenon that mathematicians may use 
an idea spontaneously, without being conscious of it in a more abstract setting. 

\citeauthor{Duality} \parencite*{Duality} also opened the way to the study of categories with additional 
structure, which some years later developed to the study of abelian categories. This 
topic was developed---in a remarkably short time---due to the work of Alexandre 
Grothendieck, David Buchsbaum, Pierre Gabriel, Max Kelly and authors of two 
monographs: Peter  \citeauthor{Freyd} \parencite*{Freyd} and Barry \citeauthor{Mitchell} \parencite*{Mitchell}. 

Within 20 years CT, originally conceived as a useful language for mathematicians, 
became a developed, mature theory, something totally unexpected by its founders. 

\subsection{Set theory without elements} 
The results of the work of William Lawvere turned out to be not only a new branch of 
CT, but also opened new perspectives in mathematics, logic, foundations of mathematics, 
and philosophy. In his Ph.D.\ thesis at Columbia University in New York, 
supervised by Eilenberg (defended in 1963, known from various copies, with full 
text published 40 years later) many new ideas were presented, including a 
categorical approach to algebraic theories \parencite{Law-Semant}. 

Lawvere also tackled the general question as to what conditions a category 
must satisfy in order to be equivalent to the category \textbf{Set}. The idea looked 
analogous to the so-called \textit{representation theorems}, i.e., propositions 
asserting that any model of the axioms for a certain abstract structure must be 
(in some prescribed sense) isomorphic to a specific type of models of the theory 
or to one particular concrete model.\footnote{The oldest theorems of this type 
are: Cayley's theorem that every (abstract) group is isomorphic to a group of 
bijections of a set; Kuratowski's theorem that every partially ordered set is  
order-isomorphic to a family of subsets of a set, ordered by inclusion; theorem that 
every group with one free generator is isomorphic to $\mathbb{Z}$. 
Analogous examples are known in many theories.} % end footnote 
However, Lawvere's case was unique and controversial in the sense that his `sets' 
were conceived \textit{without elements}. The theory did not have the primitive 
notion ``element of''. And it did work. 

Specifically, Lawvere characterized \textbf{Set} (up to \textit{equivalence} of 
categories) as a category $\mathcal{C}$ with the following: an \textit{initial} 
object \textbf{0}; a \textit{terminal} object \textbf{1} (which gives rise to 
\textit{elements} of $A$ defined as morphisms from \textbf{1} to $A$); 
\textit{products} and \textit{coproducts} of finite families of objects; 
\textit{equalizers} and \textit{coequalizers}; for any two objects there is an 
\textit{exponential}; existence of a specific object \textbf{N} with morphisms 
$0\colon \mathbf{1} \to \mathbf{N}$ and $s\colon \mathbf{N}\to \mathbf{N}$ yielding 
the successor operation $s$ on $\mathbf{N}$ and a simple recursion for sequences; 
axiom that \textbf{1} is a \textit{generator} (if parallel morphisms $f, g$ are not 
equal, then there is an element $x\in A$ such that $xf\ne xg$); axiom of choice; 
three additional elementary axioms of this sort (everything in the language of CT). 
This was augmented with one non-elementary axiom: $\mathcal{C}$ has products and 
coproducts for any \textit{indexing infinite set}. A coproduct of copies of 
\textbf{1} played the role of a set \parencites[]{Law-Sets}[][pp.386--407]{Form}[][pp.341--345]{Century}.  

The point was not to avoid membership relation completely, but (instead of 
taking as the starting point the primitive notions of \textit{set}, 
\textit{element} and membership $\in $) one takes \textit{function} as a primitive 
notion of the theory (with suitable axioms, using elementary logic, but avoiding any 
reference to sets) and then one derives membership and most concepts of set theory 
as a special case from there. 

In the second half of the 1960's Lawvere opened a way to a new theory of \textit{elementary 
toposes} (called also \textit{elementary topoi}, with Greek plural 
$\tau{\acute{o}}\pi o\iota$ of the noun $\tau{\acute{o}}\pi o\varsigma$). 
Unexpected territories of mathematics were discovered \parencites[][]{Law-Toposes}[][pp.352--359]{Century}[][]{Kromer}. 

CT became a contender for a foundation of mathematics, although the hope that it 
undermine the overwhelming role of set theory turned out spurious and most 
working mathematicians keep away from CT and toposes. CT yields new tools to study 
many formal mathematical theories and mutual relations between them, from a 
perspective different from that set theory.  

\section{Recapitulation of some points}  
Let us recall Atiyah's remark (quoted in the Introduction) that really important 
discoveries get later omitted altogether as they become absorbed by the general 
mathematical culture. This thought fits particularly well with the case of CT. Most 
of the ideas presented by Eilenberg and Mac Lane in 1945 have been absorbed as a 
natural language of advanced mathematical thinking. Once mathematicians learnt the 
definitions of a functor and a natural transformation, these concepts became a 
major tool of mathematical thinking in many abstract theories of the second half 
of the 20th century. However, it took several years to realise the scope of the 
change. Freyd commented as follows: 

\myquote{
MacLane's definition of ``product'' \parencite*{Duality} as the solution of a universal 
mapping problem was revolutionary. So revolutionary that it was not immediately 
absorbed even by most category minded people. \par 
[...] In a new subject it is often very difficult to decide what is trivial, 
what is obvious, what is hard, what is worth bragging about \parencite[p.156]{Freyd}. 
}

\noindent Mac Lane, however, used the definition of a product and its dual only in 
the case of groups (general or abelian). He did not formulate it \textit{mutatis 
mutandis} in the general case of a category, although he had several simple examples 
at hand. 
Freyd also told the story of the term \textit{exact sequence}, a technical 
definition in homological algebra. In the late 1950's, when he was a graduate student 
at Brown University, he was brought up to think in terms of exactness of maps. 
This concept seemed to him as fundamental as the notion of continuity must seem 
to an analyst. And later he was astonished to hear that when Eilenberg and  
Steenrod wrote their fundamental book \parencite{Steenrod} (published in 1952) they 
defined this very notion, recognized the importance of the choice of a suitable 
name for it, and could not invent any satisfactory word. Consequently, they 
wrote the word ``blank'' throughout most of the manuscript, ready to replace it 
before submitting the book for publication. After entertaining an unrecorded number 
of possibilities they settled on ``exact'' \parencite[p.157]{Freyd}. 

One may argue that the 1945 definitions of a category and of a functor were within a 
major onward development of abstract algebra and other advanced topics. In fact, 
originally they were not regarded as a novelty. Eilenberg and Mac Lane were not even 
certain whether their paper will be accepted for publication (it was long and lacked 
theorems with substantial proofs). However, they were genuinely convinced of the 
significance of their conceptual clarification and took pains to write the paper 
clearly and to attract the reader. 

After this publication for almost ten years CT appeared dormant. The groundbreaking 
papers on abelian categories by Buchsbaum and Gro\-then\-dieck marked 
a far-reaching change. And then---in the 1960's---CT unexpectedly started to 
grow rapidly, with astonishing results \parencite[pp.338--339, 341--361]{Century}.

Thus, from the present perspective, in spite of the previous arguments, one can say 
that the emergence of CT was undoubtedly a major transgression in mathematics. 
It was a crossing of a previously non-traversable barrier of deep-rooted habits to 
think of mathematics. A vivid argument is the fact that---even after publication 
of the main ideas---it was so difficult to overcome the previous inhibition 
and widespread tradition.\footnote{Many mathematicians, in USA and elsewhere, 
expressed disinclination to CT. Karol Borsuk, an outstanding topologist, the 
teacher of Eilenberg in Warsaw and coauthor of their joint paper published in 1936, 
was later unfavourable to CT and the categorical methods in mathematics 
\parencite[p.30]{Jackowski}. Jerzy Dydak, a student of Borsuk, recalled after years: 
\textit{My own PhD thesis written under Borsuk in 1975 makes extensive use 
of category theory and I was asked by him to cut that stuff out. Only after 
I assured him that I spent many months trying to avoid abstract concepts, he 
relinquished and the thesis was unchanged} \parencite[p.92]{Dydak}.} % end footnote

The creators of CT and their followers could choose their definitions freely, nobody 
could forbid that. And yet the previous way of thinking was an obstacle for 
potential authors and for prospective readers. Great insight of Eilenberg and 
Mac Lane of what is significant in mathematics turned out a crucial factor.  



\end{artengenv}