\begin{editorialeng}{Michał Eckstein, Bartłomiej Skowron}
	{``Is logic a~physical variable?'' Introduction to the Special Issue}
	{``Is logic a~physical variable?'' Introduction to the Special Issue}
	{``Is logic a~physical variable?'' Introduction to the Special Issue}
	%	{Copernicus Center for Interdisciplinary Studies}
	{Philosophy of economics -- a~school of pluralism and humility}
	%	{Abstrakt lorem ipsum}
	%	{słowo, słowo.}
	
	


\lettrine[loversize=0.13,lines=2,lraise=-0.03,nindent=0em,findent=0.2pt]%
{``I}{}s logic a~physical variable?'' This thought-provoking question was put forward by Michael Heller during the public lecture ``Category Theory and Mathematical Structures of the Universe'' delivered on 30\textsuperscript{th} March 2017 at the National Quantum Information Center in Sopot. It touches upon the intimate relationship between the foundations of physics, mathematics and philosophy. To address this question one needs a~conceptual framework, which is on the one hand rigorous and, on the other hand. capacious enough to grasp the diversity of modern theoretical physics. Category theory is here a~natural choice. It is not only an independent, well-developed and very advanced mathematical theory, but also a~holistic, process-oriented way of thinking.

Michael Heller's inspiring question provided the first impulse to organize a~meeting of mathematicians, physicists and philosophers interested in exploring the applications of category theory in their research. The actual conference, co-organized by the Copernicus Center for Interdisciplinary Studies and the International Center for Formal Ontology took place on 7-9 November 2019 in Kraków, Poland. The Conference was the 23\textsuperscript{rd} edition of Kraków Methodological Conferences, held regularly since 1992, devoted to fundamental problems at the junction between philosophy and sciences. The conference proceedings are collected in this volume.

The papers in the volume have been authored by philosophers, physicists and mathematicians. The common denominator uniting these works is a~\textit{categorical cognitive perspective}, i.e. category theory, which is gaining more and more interest in Poland, is the central axis of the volume.

Category theory was developed in the early 1940s. It resulted from the combination of topological talent of the Polish-American mathematician Samuel Eilenberg and the algebraic talent of the American mathematician Saunders Mac Lane. In the sixties, the theory developed significantly, becoming a~fully-fledged and independent branch of mathematics. The development of category theory often took place in directions previously unanticipated by its creators. In particular, the deep insights by William Lawvere's have shown that category theory has enough potential to even shake the very foundations of mathematics. Today, category theory is a~standard tool not only for mathematicians, but also theoretical physicists and philosophers. Category theory is full of unexpected relationships, deep analogies and hidden connections between seemingly unrelated structures, arousing interest among many scientists and scholars, thus reaching far beyond pure mathematics.

Category theory implies a~formal ontology that emphasizes the relationship between objects and thus diminishes the role of the objects themselves. This aspect attracts many scholars, but for many it also makes it difficult to understand what it is all about. Thus category theory requires a~certain effort to change the object-oriented cognitive attitudes. Below we briefly present the content of the all articles published in the volume to help the Reader to discover the various aspects of category theory discussed in this volume.

Colin McLarty in the paper \textit{Mathematics as a~love of wisdom: Saunders Mac Lane as philosopher} points out the similarities between Aristotle's and Saunders Mac Lane's thoughts. The author describes this unexpected analogy in a~very interesting way. Mac Lane, together with Eilenberg, have established category theory. They had a~great influence on the development of mathematics in the 20\textit{\textsuperscript{th}} century. From the beginning of his career Mac Lane was interested in philosophy, nevertheless, not as an academic philosopher, but just as a~working mathematician. McLarty points out that both Aristotle and Mac Lane treated ``knowledge'' as the ``knowledge of reasons''. For Mac Lane, mathematical understanding is not only to know the proof of a~given theorem, but also to know the \textit{reasons} for that theorem.

According to George Cantor ``The very essence of mathematics lies precisely in its freedom''. We can understand this statement in such a~way that a~mathematician can introduce into mathematics the concepts (s)he wants to introduce, there are no restrictions in this process, except the logical constraints. This is very attractive and this is what makes us want to practice pure mathematics! Zbigniew Semadeni in his paper \textit{Creating new concepts in mathematics: freedom and limitations. The case of Category Theory} ponders the question: ``What influenced the emergence of category theory?'', from the perspective of Platonizing constructivism. The author claims that despite the fact that the emergence of CT was a~transgression (i.e. \textit{crossing of a~previously non-traversable limit of mathematical knowledge}), the development of the notion of function since the beginning of the 19\textsuperscript{th} century was one of the important factors in the emergence of CT. The author also analyzes the origin of the term \textit{functor}, claiming that it can be found already in the works of Tarski and Kotarbiński.

Structuralism in the philosophy of mathematics is a~position, painting it with a~broad brush, according to which mathematics concerns certain abstract structures. It seems natural to ask what logic is behind a~given version of structuralism? Logic can influence the properties of given structures, e.g. metalogical properties. Is it first-order logic, higher-order logic, or maybe modal logic, or even other? Jean-Pierre Marquis in his paper \textit{Abstract logical structuralism} insightfully points out that, first of all, structuralism is also about logic itself: that is, structuralism should also be interested in pondering what kind of mathematical structure hides behind logic itself and what are the relations of that structure to other mathematical structures. Secondly, and more importantly, the author indicates and convincingly justifies, giving many examples of mathematical structures related to logic, that this is in fact a~categorical logic. In the words of the author: ``Indeed, this is what categorical logic is all about: it reveals the abstract mathematical structures of logic and it relates them to other abstract mathematical structures, revealing yet other structural features''.

The theory of toposes is typically linked with intuitionistic logic, although the actual connection is more subtle. In 1995 Chris Mortensen formulated the concept of a~co-topos, based on closed rather than open sets, which is related to a~paraconsistent logic. In his article \textit{On the validity of the definition of a~compelement-classifier} Mariusz Stopa has a~critical take on co-toposes. He analyses the definition of the ``true'' arrow and argues that the definition cannot be interpreted arbitrarily. As Stopa acknowledges, his analysis hinges upon the equivalence of two definitions of toposes provided by Mac Lane and Moerdijk in 1994.

Toposes are also at the core of the article \textit{No-signaling in topos formulation and a~common ontological basis for classical and non-classical physical theories} by Marek Kuś. The author provides a~concise introduction to Bell-like inequalities, which triggered the foundational tests of quantum mechanics against classical, hidden variables, theories. His analysis extends also to ‘post-quantum' scenarios involving the so-called no-signalling boxes introduced by Popescu and Rohrlich in 1994. The article explains the internal logic of classical, quantum and no-signalling theories. Kuś also argues that, with the help of toposes, one can unveil a~common ontological basis for all three theories.

In 1935 Niels Bohr, in response to the now famous paradox unravelled by Einstein, Podolsky and Rosen, has written: ``The apparent contradiction in this fact [i.e. the impossibility of attaching definite values to both of two canonically conjugated variables] discloses only an essential inadequacy of the customary viewpoint of natural philosophy for a~rational account of physical phenomena of the type with which we are concerned in quantum mechanics.'' In his article \textit{Quantum contextuality as a~topological property, and the ontology of potentiality} Marek Woszczek takes this viewpoint and argues that quantum theory challenges our basic intuitions on ``physical definitness'' and ``objective reality'' and calls for an ``ontology of potentiality''. His pivotal argument is based on the quantum contextuality, which could be seen---in a~category theory spirit---as a~topological property.

\textit{Quantum geometry, logic and probability} is the title of the article by Shahn Majid. The author provides an inviting introduction to his original formalism of discrete quantum geometry based on graphs. The main result is a~surprising connection between Markov processes and a~generalized form of (noncommutative) Riemannian geometry. Then, Majid introduces a~{\textasciigrave}discrete Schrödinger process' and shows that it induces a~Markov-like correction to the standard continuity equation in quantum mechanics. In the last part, the author argues in favour of a~(quantum) geometric viewpoint on de Morgan duality and discusses its consequences and potential application in the quest of constructing a~consistent theory of quantum gravity.

In today's world filled with virtual objects, information plays an incomparably greater role than before, when there were no computers and so many objects generated by computers. In the text \textit{Information and physics} Radosław Kycia and Agnieszka Niemczynowcz present, in a~way that is accessible to non-specialists, studies that shed light on the relationship between the abstract image of information on the one hand and its physical (or ``real'') representative on the other. The authors present a~solution to the Maxwell paradox using Landauer's principle. This principle links in a~non-obvious manner the computational processes to the production of measurable physical consequences, i.e. to the generation of heat. In their presentation the authors refer to their research on the general method of building thermodynamic analogues of computer memory. In this method, the authors find Galois connection, which makes category theory play a~key role in this research.

The mind–body dichotomy is one of the most profound philosophical problems. Although the physical structure of the brain, along with the basic mechanisms of its functioning, is systematically unravelled by modern neurobiology, the problem of how does the brain (or, rather, mind) ascribe meaning to the processed data is unsolved. Steve Awodey and Michael Heller in their though-provoking article \textit{The homunculus brain and categorical logic} take inspiration from the intricate duality between syntax and semantics, well established within category theory. They propose a~toy-model---a~``homunculus brain''---with neurons modeled by categories and axons by functors. The homunculus mind, is, on the other hand, modeled by a~category of theories. The authors introduce the BRAIN and MIND categories and show - using the results from categorical logic, including the syntax-semantics relationship - how the structure enhanced by adjoint functors ``Lang'' and ``Syn'' enable the creation of meanings. As the authors summarize in a~typical categorical attitude of finding unobvious, deep and far-reaching connections: ``The categories BRAIN and MIND interact with each other with their entire structures and, at the same time, these very structures are shaped by this interaction''.

As the guest editors of this volume we would like to express our sincere gratitude to all of the authors for their valuable contributions, to the anonymous reviewers for detailed and instructive reviews, and to the editor-in-chief of \textit{Philosophical Problems in Science}, Paweł Polak, for his kind support of this initiative. We are also very grateful to the editorial secretary Piotr Urbańczyk for his commitment at almost every stage of the preparation of this volume and to Roman Krzanowski for the proofreading.

We acknowledge the financial support of the Minister of Science and Higher Education under the programme for dissemination of science [761/P-DUN/2019] towards the organization of the 23\textsuperscript{rd} Kraków Methodological Conference.


\end{editorialeng}